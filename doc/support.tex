\section{Reasoning support}

We now turn to the second part of the \Equations package: the derivation
of support definitions to help reason on the generated implementations.

\subsection{Building equations}

The easiest step is constructing the proofs of the equations as
propositional equalities. We recurse on the splitting tree, 
bookkeeping the name of the current function \cst{f}
and for each $\Compute{Δ \vdash \vec{p} : Γ}{rhs}$
node we inspect the right-hand side and generate a statement:
\begin{itemize}
\item $\Program{t}$: the equation is simply $Π~Δ, \cst{f}~\vec{p}~=t$.
\item $\Empty{t}$: we make an instance of the following typeclass:
  \input{imposs.coq}
  In this case we generate an instance of
  $Π~Δ, \class{ImpossibleCall} (\cst{f}~\vec{p})$.
\item $\Refine{t}{c'}{\cst{f'}}{s}$: we generate an indirection equation for the
  helper function:
  $Π~Δ, \cst{f}~\vec{p} = \cst{f'}  $

\end{itemize}





 that is solvable by unfolding 

\subsection{Induction principle}


%%% Local Variables: 
%%% mode: latex
%%% TeX-PDF-mode: t
%%% TeX-master: "equations"
%%% End: 