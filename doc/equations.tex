\documentclass{report}
\usepackage[T1]{fontenc}
%\usepackage{fullpage}
\usepackage{amsmath,amssymb}
\usepackage{qsymbols}
\usepackage{code}
\usepackage{coq}
\usepackage[color]{coqdoc}
\usepackage{hyperref}
\usepackage{abbrevs}
\usepackage{natbib}
\usepackage{minitoc}
\usepackage[utf8x]{inputenc}

% \makeatletter
% \renewcommand{\tableofcontents}{\noindent\@starttoc{toc}}
% \makeatother
\def\texttau{\ensuremath{\tau}}
\def\textDelta{\ensuremath{\Delta}}
\def\textGamma{\ensuremath{\Gamma}}

\setlength{\coqdocbaseindent}{1em}

\def\eqnversion{\texttt{1.3beta2}\xspace}

\author{Matthieu Sozeau}
\date{\today}
\title{\Equations \eqnversion Reference Manual}
\begin{document}

\maketitle

\def\coqlibrary#1#2#3{}

\def\Equations{\texorpdfstring{\name{Equations}}{Equations}}

\chapter*{Introduction}
\label{cha:introduction}

\Equations is a toolbox built as a plugin on top of the \Coq proof assistant
to program and reason on programs defined by full \emph{dependent}
pattern-matching and \emph{well-founded} recursion. While the primitive
core calculus of \Coq allows definitions by \emph{simple} pattern-matching on
inductive families and \emph{structural} recursion, \Equations extends
the set of easily definable constants by allowing a richer form of
pattern-matching and arbitrarily complex recursion schemes. It can be
thought of as a twin of the \name{Function} package for Isabelle that
implements a \emph{definitional} translation from partial, well-founded
recursive functions to the HOL core logic. See
\cite{BoveKraussSozeau2011} for an overview of tools for defining
recursive definitions in interactive proof assistants like (\Coq, \Agda
or \Isabelle).

The first version of the tool was described in
\cite{sozeau.Coq/Equations/ITP10}, the most recent one is described in
\cite{equationsreloaded}. This manual provides a documentation of the plugin
commands (\autoref{cha:manual}) followed by a tutorial using basic
examples (\autoref{cha:gentle-intro}).  More elaborate examples are
available at \url{http://mattam82.github.io/Coq-Equations/examples}.

This manual describes version \eqnversion of the package.

\section*{Installation}

Equations is available through the
\texttt{opam}\footnote{\url{http://opam.ocaml.org}} package manager as
package \texttt{coq-equations}. To install it on an already existing
\texttt{opam} installation with the \Coq repository, simply input the
command:
\begin{verbatim}
# opam install coq-equations
\end{verbatim}

If you want to use it with the \href{https://github.com/HoTT/HoTT}{HoTT library for Coq}, simply install the 
\texttt{coq-hott} package before \Equations.

The development version and detailed installation instructions are available at
\url{http://mattam82.github.io/Coq-Equations}.

\doparttoc
\parttoc
\tableofcontents

\chapter{Manual}
\label{cha:manual}
\section{The \kw{Equations} Vernacular}
\label{manual}

\subsection{Syntax of programs}
\def\kw#1{\coqdockw{#1}}
\def\vec#1{\overrightarrow{#1}}
\def\vecplus#1{{\overrightarrow{#1}}^{+}}
\def\textcoloneq{\texttt{:=}}
\def\userref#1#2{\coqdockw{with}~#1~\textcoloneq~#2}
\def\var#1{\coqdocvar{#1}}
\def\figdefs{\begin{array}{llcl}
  \texttt{term}, \texttt{type} & t, ~τ & \Coloneqq &
  \coqdocvar{x} `| \lambda \coqdocvar{x} : \tau, t, R `| ∀ \coqdocvar{x} :
                                                    \tau, \tau' `|
  \mathbf{\lambda}\texttt{\{}\,\vecplus{\vec{up} \coloneqq t}\texttt{\}}
   \cdots \\
  \texttt{binding} & d & \Coloneqq & \texttt{(}\coqdocvar{x}~\texttt{:}~\tau\texttt{)} `|
  \texttt{(}\coqdocvar{x}~\textcoloneq~t~\texttt{:}~\tau\texttt{)} \\
  \texttt{context} & Γ, Δ & \Coloneqq & \vec{d} \\
  \texttt{programs} & progs & \Coloneqq & prog~\overrightarrow{mutual} \texttt{.} \\
  \texttt{mutual programs} & mutual & \Coloneqq & \coqdockw{with}~p `| where \\
  \texttt{where clause} & where & \Coloneqq & \coqdockw{where}~p `| \coqdockw{where}~not\\
  \texttt{notation} & not & \Coloneqq & \texttt{''}string\texttt{''}~\textcoloneq~t~(\texttt{:}~scope)?\\
  \texttt{program} & p, prog & \Coloneqq & \coqdoccst{f}~Γ~\texttt{:}~τ~(\coqdockw{by}~\textit{annot})?~\textcoloneq~clauses \\
  \texttt{annotation} & annot & \Coloneqq & \kw{struct}~\var{x}? `| \kw{wf}~t~R? \\
  \texttt{user clauses} & clauses & \Coloneqq & \vecplus{cl} `| \texttt{\{}\,\vec{cl}\,\texttt{\}} \\
  \texttt{user clause} & cl & \Coloneqq & \coqdoccst{f}~\vec{up}~n?~\texttt{;} `|
                                         \vecplus{\texttt{|}~up}~n?
                                         ~\texttt{;} \\
  \texttt{user pattern} & up & \Coloneqq & x `| \_
  `| \coqdocconstr{C}~\vec{up}
  `| \texttt{?(}\,t\,\texttt{)} `| \texttt{!} \\
  \texttt{user node} & n & \Coloneqq &
   \textcoloneq~t~\overrightarrow{where} `|\, \userref{t~\vec{, t}}{clauses}
\end{array}}

In the grammar, $\vec{t}$ denotes a possibly empty list of $t$,
$\vecplus{t}$ a non-empty list. Concrete syntax is in
\texttt{typewriter} font.
\begin{figure}[h]
\centering$\figdefs$
\caption{Definitions and user clauses}
\label{fig:usergram}
\end{figure}
The syntax allows the definition of toplevel mutual (\kw{with}) and
nested (\kw{where}) structurally recursive definitions. Notations can be
used globally to attach a syntax to a recursive definition, or locally
inside a user node. A single program is given as a tuple of a (globally
fresh) identifier, a signature and a list of user clauses (order
matters), along with an optional recursion annotation (see next
section). The signature is simply a list of bindings and a result
type. The expected type of the function \cst{f} is then $∀~Γ, τ$.
An empty set of clauses denotes that one of the variables has an empty type.

Each user clause comprises a list of patterns that will match the
bindings $Γ$ and an optional right hand side. Patterns can be named or
anonymous variables, constructors applied to patterns, the inaccessible
pattern \texttt{?(}t\texttt{)} (a.k.a. "dot" pattern in \Agda) or the
empty pattern \texttt{!} indicating a variable has empty type (in this
case only, the right hand side must be absent). Patterns are parsed
using \Coq's regular term parser, so any term with implicit arguments
and notations which desugars to this syntax is also allowed.

A right hand side can either be a program node returning a term $t$
potentially relying on auxiliary definitions through local \kw{where}
clauses, or a \kw{with} node.  Local \kw{where} clauses can be used to
define nested programs, as in \Haskell or \Agda, or local
notations. They depend on the lexical scope of the enclosing program. As
programs, they can be recursive definitions themselves and depend on
previous \kw{where} clauses as well: they will be elaborated to
dependent let bindings. The syntax permits the use of curly braces
around a list of clauses to allow disambiguation of the scope of
\kw{where} and \kw{with} clauses. The $\lambda\{$ syntax (using a
unicode lambda attached to a curly brace) extends \Coq's term syntax
with pattern-matching lambdas, which are elaborated to local \kw{where}
clauses. A local \kw{with}~$t$ node essentialy desugars to a program
node with a local \kw{where} clause taking all the enclosing context as
arguments plus a new argument for the term $t$, and whose clauses are
the clauses of the \kw{with}. The \kw{with} construct can be nested also
by giving multiple terms, in which case the clauses should refine a new
problem with as many new patterns.

\subsection{Generated definitions}

Upon the completion of an \Equations definition, a few supporting lemmas
are generated.

\subsubsection{Equations}

Each compiled clause of the program or one
of its subprograms defined implicitely by \kw{with} or explicitely by
\kw{where} nodes gives rise to an equation. Note that the clauses
correspond to the program's splitting tree, i.e. to the expansion of
pattern-matchings, so a single source clause catching multiple cases
can correspond to multiple equations. All of these equations are
registered as hints in a rewrite hint database named $\cst{f}$, which can be
used by the \coqdoctac{simp} or \coqdoctac{autorewrite} tactic
afterwards. The $\coqdoctac{simp}~f$ tactic is just an alias to
$\coqdoctac{autorewrite with}~f$. The equation lemmas are named
after the position they appear in in the program, and are of the
form $\cst{f}\_clause\_n\_equation\_k$.

In case the program is well-founded, \Equations first generates an
unfolded definition named \cst{f\_unfold} corresponding to the
1-unfolding of the recursive definition and shows that it is
extensionally equal to \cst{f}. This unfolding equation is used
to generate the equations associated to \cst{f}, which might also
refer to the unfolded versions of subprograms. Well-founded
recursive definitions can hence generate a set of equations that
is not terminating as an unconditional rewrite system.

\subsubsection{Elimination principle}

\Equations also automatically generates a mutually-inductive relation
corresponding to the graph of the programs, whose first inductive is named
$\cst{f}\_ind$. It automatically shows that the functions respects their
graphs (lemma $\cst{f}\_ind\_fun$) and derives from this proof an
elimination principle named $\cst{f}\_elim$. This eliminator can be used
directly using the \tac{apply} tactic to prove goals involving a call to
the function(s). One has to provide predicates for each of the toplevel
programs and the \kw{where} subprograms (\kw{with} subprograms's
predicates follow from their enclosing programs).

In case the program has a single predicate, one can use the
$\tac{funelim}~call$ tactic to launch the elimination by specifying
which call of the goal should be used as the elimination target.
A most general predicate is inferred in this case.

\subsection{Local Options}
The \kw{Equations} command takes a few options using the syntax
\[\kw{Equations}(opts)~\cst{f}~\ldots\]

\begin{itemize}
\item \texttt{noind}: Do not generate the inductive graph of the
  function and the derived eliminator.
\item \texttt{noeqns}: Do not generate the equations correponding to the
  (expanded) clauses of the program. This implies \texttt{noind}.
\end{itemize}

One can use the \kw{Equations?} syntax to use the interactive proof mode
instead of obligations to resolve holes in the term or obligations
comming from well-founded recursive definitions. BEWARE that the use
of the \texttt{abstract} tactical is not well-supported in this mode.

\subsection{Global Options}

The \kw{Equations} command obeys a few global options:
\begin{itemize}
\item \texttt{Equations Transparent}: governs the opacity of definitions
  generated by \kw{Equations}. By default this is off and means that
  definitions are declared \emph{opaque} for reduction, avoiding
  spurious unfoldings when using the \texttt{simpl} tactic for example.
  The \texttt{simp} $\cst{c}$ tactic is favored in this case to do
  simplifications using the equations generated for $\cst{c}$.

\item \texttt{Equations With Funext}: governs the use of the functional
  extensionality axiom to prove the unfolding lemma of well-founded
  definitions, which requires extensionality of the functional. By
  default \emph{on}. When this flag is off, the unfolding lemmas of
  well-founded definitions might fail to be proven automatically and be
  left to the user as an obligation. To prove this obligation, the user
  is encouraged to use the \texttt{Equations.Init.unfold_recursor}
  tactic to help solve goals of the form \[\cst{FixWf} x f = \cst{f\_unfold} x\].

\item \texttt{Equations With UIP}: governs the use of instances of
  \texttt{Equations.Classes.UIP} derived by the user, or automatically
  from instances of the decidable equality class
  \texttt{Equations.Classes.EqDec}. By default \emph{off}. When switched
  on, equations will look for an instance of $\ind{UIP}\~A$ when solving
  equalities of the form \[\forall (e : x = x :> A), P e\], i.e. to
  apply the deletion rule to such equations, or to unify indices of
  constructors for inductive families without a \ind{NoConfusionHom}
  instance. It will report an error if it cannot find any. Note that
  when this option is on, the computational behavior of \Equations
  definitions on open terms does not follow the clauses: it might block
  on the uip proof (for example if it is a decidable equality test).
  The rewriting equations and functional elimination principle can still
  be derived though and are the prefered way to reason on the
  definition.

\item \texttt{Equations WithK} DEPRECATED. Use \texttt{With UIP} and
  declare your own version of the \cst{UIP} axiom as a typeclass
  instance. Governs the use of the \texttt{K} axiom.
  By default \emph{off}. The computational behavior of definitions
  using axioms changes entirely: their reduction will get stuck even
  on closed terms. It is advised to keep such definitions opaque and use
  the derived rewriting equations and functional elimination principle
  to reason on them.
\end{itemize}

\section{Derive}

\Equations comes with a suite of deriving commands that take inductive
families and generate definitions based on them. The common syntax for
these is:

\[\mathtt{Derive}~\ind{C}_1 \ldots \ind{C}_n~\mathtt{for}~\ind{ind}_1 \ldots \ind{ind}_n.\]

Which will try to generate an instance of type class \ind{C} on
inductive type \ind{Ind}. We assume $\ind{ind}_i : \Pi \Delta. s$.
The derivations provided by \Equations are:

\begin{itemize}
\item \ind{DependentEliminationPackage}: generates the dependent
  elimination principle for the given inductive type, which can differ
  from the standard one generated by \Coq.
  It derives an instance of the class

  \texttt{Equations.DepElim.DependentEliminationPackage}.
\item \ind{Signature}: generate the signature of the inductive, as a
  sigma type packing the indices $\Delta$ (again as a sigma type) and
  an object of the inductive type. This is used to produce homogeneous
  constructions on inductive families, by working on their packed
  version (total space in HoTT lingo).
  It derives an instances of the class
  \texttt{Equations.Signature.Signature}.

\item \ind{NoConfusion}: generate the no-confusion principle for the
  given family, as an heterogeneous relation. It embodies the
  discrimination and injectivity principles for the
  total space of the given inductive family: i.e.
  $\Sigma \Delta, \ind{I}~\bar{\Gamma}~\Delta$ for a family
  $\ind{I} : \forall \Gamma, \Delta "->" \kw{Type}$ where $\Gamma$ are
  (uniform) parameters of the inductive and $\Delta$ its indices.

  It derives an instance of the class \texttt{Equations.DepElim.NoConfusionPackage}.

\item \ind{NoConfusionHom}: generate the \emph{homogeneous} no-confusion
  principle for the given family, which embodies the discrimination and
  injectivity principles for (non-propositional) inductive types.
  This principle can be derived if and only if the no-confusion property
  on the inductive family instance reduces to equality of the non-forced
  arguments of the constructors. In case of success it generates an instance of the class
  \texttt{Equations.DepElim.NoConfusionPackage} for the type $\ind{I}~
  \Delta~\Gamma$ applicable to equalities of two objects in the \emph{same}
  instance of the family $\ind{I}$.

\item \ind{EqDec}
  This derives a decidable equality on $C$, assuming decidable equality 
  instances for the parameters and supposing any primitive inductive
  type used in the definition also has decidable equality. If
  successful it generates an instance of the class (in \texttt{Equations.Classes.EqDec}):
\begin{verbatim}
Class EqDec (A : Type) :=
  eq_dec : forall x y : A, { x = y } + { x <> y }.
\end{verbatim}
  
\item \ind{Subterm}: this generates the direct subterm relation for the
  inductive (asuming it is in \kw{Set} or \kw{Type}) as an inductive family.
  It then derives the well-foundedness of this relation and wraps it
  as an homogeneous relation on the signature of the datatype (in case
  it is indexed). These relations can be used with the \texttt{by wf}
  clause of equations. It derives an instance of the class
  \texttt{Equations.Classes.WellFounded}.

\end{itemize}

\section{\texttt{dependent elimination}}

The \texttt{dependent elimination} tactic can be used to do dependent
pattern-matching during a proof, using the same engine as Equations.

Its syntax is:
\begin{figure}[h]
  \texttt{dependent elimination} \textit{ident} \texttt{as} [ up | .. | up ].
\end{figure}

It takes a list of patterns (see figure \ref{fig:usergram}) that should cover the type of \textit{ident}
and generates the corresponding subgoals.

%%% Local Variables: 
%%% mode: latex
%%% TeX-master: "equations"
%%% TeX-PDF-mode: t
%%% End: 


\chapter{A gentle introduction to Equations}
\label{cha:gentle-intro}

The source of this chapter that can be run in Coq with Equations
installed is available at:

\url{https://raw.githubusercontent.com/mattam82/Coq-Equations/master/doc/equations_intro.v}

\coqlibrary{equations intro}{Library }{equations\_intro}

\begin{coqdoccode}
\coqdocemptyline
\end{coqdoccode}
\coqdockw{Equations} is a plugin for \cite{Coq} that comes with a few support modules defining
   classes and tactics for running it. We will introduce its main
   features through a handful of examples. We start our Coq primer
   session by importing the \coqdockw{Equations} module.  \begin{coqdoccode}
\coqdocemptyline
\coqdocnoindent
\coqdockw{Require} \coqdockw{Import} \coqexternalref{}{http://coq.inria.fr/distrib/V8.10.1/stdlib//Coq.Arith.Arith}{\coqdoclibrary{Arith}} \coqexternalref{}{http://coq.inria.fr/distrib/V8.10.1/stdlib//Coq.omega.Omega}{\coqdoclibrary{Omega}} \coqexternalref{}{http://coq.inria.fr/distrib/V8.10.1/stdlib//Coq.Program.Program}{\coqdoclibrary{Program}}.\coqdoceol
\coqdocnoindent
\coqdockw{From} \coqdockw{Equations} \coqdockw{Require} \coqdockw{Import} \coqdoclibrary{Equations}.\coqdoceol
\coqdocemptyline
\coqdocemptyline
\end{coqdoccode}
\section{Inductive types}




   In its simplest form, \coqdockw{Equations} allows to define functions on inductive datatypes.
   Take for example the booleans defined as an inductive type with two constructors \coqexternalref{true}{http://coq.inria.fr/distrib/V8.10.1/stdlib//Coq.Init.Datatypes}{\coqdocconstructor{true}} and \coqexternalref{false}{http://coq.inria.fr/distrib/V8.10.1/stdlib//Coq.Init.Datatypes}{\coqdocconstructor{false}}:
   \coqdoceol
\coqdocemptyline
\coqdocindent{1.50em}
\coqdockw{Inductive} \coqexternalref{bool}{http://coq.inria.fr/distrib/V8.10.1/stdlib//Coq.Init.Datatypes}{\coqdocinductive{bool}} : \coqdockw{Set} := \coqexternalref{true}{http://coq.inria.fr/distrib/V8.10.1/stdlib//Coq.Init.Datatypes}{\coqdocconstructor{true}} : \coqexternalref{bool}{http://coq.inria.fr/distrib/V8.10.1/stdlib//Coq.Init.Datatypes}{\coqdocinductive{bool}} \ensuremath{|} \coqexternalref{false}{http://coq.inria.fr/distrib/V8.10.1/stdlib//Coq.Init.Datatypes}{\coqdocconstructor{false}} : \coqexternalref{bool}{http://coq.inria.fr/distrib/V8.10.1/stdlib//Coq.Init.Datatypes}{\coqdocinductive{bool}} 

\coqdocemptyline


   We can define the boolean negation as follows: \begin{coqdoccode}
\coqdocemptyline
\coqdocnoindent
\coqdockw{Equations} \coqdef{equations intro.neg}{neg}{\coqdocdefinition{neg}} (\coqdocvar{b} : \coqexternalref{bool}{http://coq.inria.fr/distrib/V8.10.1/stdlib//Coq.Init.Datatypes}{\coqdocinductive{bool}}) : \coqexternalref{bool}{http://coq.inria.fr/distrib/V8.10.1/stdlib//Coq.Init.Datatypes}{\coqdocinductive{bool}} :=\coqdoceol
\coqdocnoindent
\coqref{equations intro.neg}{\coqdocdefinition{neg}} \coqexternalref{true}{http://coq.inria.fr/distrib/V8.10.1/stdlib//Coq.Init.Datatypes}{\coqdocconstructor{true}} := \coqexternalref{false}{http://coq.inria.fr/distrib/V8.10.1/stdlib//Coq.Init.Datatypes}{\coqdocconstructor{false}} ;\coqdoceol
\coqdocnoindent
\coqref{equations intro.neg}{\coqdocdefinition{neg}} \coqexternalref{false}{http://coq.inria.fr/distrib/V8.10.1/stdlib//Coq.Init.Datatypes}{\coqdocconstructor{false}} := \coqexternalref{true}{http://coq.inria.fr/distrib/V8.10.1/stdlib//Coq.Init.Datatypes}{\coqdocconstructor{true}}.\coqdoceol
\coqdocemptyline
\end{coqdoccode}
\coqdockw{Equations} declarations are formed by a signature definition and a set of \textit{clauses} 
   that must form a \textit{covering} of this signature. The compiler is then expected to
   automatically find a corresponding case-splitting tree that implements the function.
   In this case, it simply needs to split on the single variable \coqdocvariable{b} to
   produce two new \textit{programming problems} \coqref{equations intro.neg}{\coqdocdefinition{neg}} \coqexternalref{true}{http://coq.inria.fr/distrib/V8.10.1/stdlib//Coq.Init.Datatypes}{\coqdocconstructor{true}} and \coqref{equations intro.neg}{\coqdocdefinition{neg}} \coqexternalref{false}{http://coq.inria.fr/distrib/V8.10.1/stdlib//Coq.Init.Datatypes}{\coqdocconstructor{false}} that are directly 
   handled by the user clauses. We will see in more complex examples that this search
   for a splitting tree may be non-trivial. 

\section{Reasoning principles}




   In the setting of a proof assistant like Coq, we need not only the ability 
   to define complex functions but also get good reasoning support for them.
   Practically, this translates to the ability to simplify applications of functions 
   appearing in the goal and to give strong enough proof principles for (recursive)
   definitions.


   \coqdockw{Equations} provides this through an automatic generation of proofs related to
   the function. Namely, each defining equation gives rise to a lemma stating the 
   equality between the left and right hand sides. These equations can be used as 
   rewrite rules for simplification during proofs, without having to rely on the
   fragile simplifications implemented by raw reduction. We can also generate the
   inductive graph of any \coqdockw{Equations} definition, giving the strongest elimination
   principle on the function. 


   I.e., for \coqref{equations intro.neg}{\coqdocdefinition{neg}} the inductive graph is defined as: \coqdoceol
\coqdocemptyline
\coqdocnoindent
\coqdockw{Inductive} \coqref{equations intro.neg graph}{\coqdocinductive{neg\_graph}} : \coqexternalref{bool}{http://coq.inria.fr/distrib/V8.10.1/stdlib//Coq.Init.Datatypes}{\coqdocinductive{bool}} \ensuremath{\rightarrow} \coqexternalref{bool}{http://coq.inria.fr/distrib/V8.10.1/stdlib//Coq.Init.Datatypes}{\coqdocinductive{bool}} \ensuremath{\rightarrow} \coqdockw{Prop} :=\coqdoceol
\coqdocnoindent
\ensuremath{|} \coqref{equations intro.neg graph equation 1}{\coqdocconstructor{neg\_graph\_equation\_1}} : \coqref{equations intro.neg graph}{\coqdocinductive{neg\_graph}} \coqexternalref{true}{http://coq.inria.fr/distrib/V8.10.1/stdlib//Coq.Init.Datatypes}{\coqdocconstructor{true}} \coqexternalref{false}{http://coq.inria.fr/distrib/V8.10.1/stdlib//Coq.Init.Datatypes}{\coqdocconstructor{false}}\coqdoceol
\coqdocnoindent
\ensuremath{|} \coqref{equations intro.neg graph equation 2}{\coqdocconstructor{neg\_graph\_equation\_2}} : \coqref{equations intro.neg graph}{\coqdocinductive{neg\_graph}} \coqexternalref{false}{http://coq.inria.fr/distrib/V8.10.1/stdlib//Coq.Init.Datatypes}{\coqdocconstructor{false}} \coqexternalref{true}{http://coq.inria.fr/distrib/V8.10.1/stdlib//Coq.Init.Datatypes}{\coqdocconstructor{true}} 

\coqdocemptyline


   Along with a proof of \coqdocvar{\ensuremath{\Pi}} \coqdocvariable{b}, \coqref{equations intro.neg graph}{\coqdocinductive{neg\_graph}} \coqdocvariable{b} (\coqref{equations intro.neg}{\coqdocdefinition{neg}} \coqdocvariable{b}), we can eliminate any call
   to \coqref{equations intro.neg}{\coqdocdefinition{neg}} specializing its argument and result in a single command. 
   Suppose we want to show that \coqref{equations intro.neg}{\coqdocdefinition{neg}} is involutive for example, our goal will 
   look like: \coqdoceol
\coqdocemptyline
\coqdocindent{1.00em}
\coqdocvariable{b} : \coqexternalref{bool}{http://coq.inria.fr/distrib/V8.10.1/stdlib//Coq.Init.Datatypes}{\coqdocinductive{bool}}\coqdoceol
\coqdocindent{1.00em}
============================\coqdoceol
\coqdocindent{1.50em}
\coqref{equations intro.neg}{\coqdocdefinition{neg}} (\coqref{equations intro.neg}{\coqdocdefinition{neg}} \coqdocvariable{b}) = \coqdocvariable{b} 

\coqdocemptyline
   An application of the tactic \coqdocvar{\coqdoctac{funelim}} (\coqref{equations intro.neg}{\coqdocdefinition{neg}} \coqdocvariable{b}) will produce two goals corresponding to 
   the splitting done in \coqref{equations intro.neg}{\coqdocdefinition{neg}}: \coqref{equations intro.neg}{\coqdocdefinition{neg}} \coqexternalref{false}{http://coq.inria.fr/distrib/V8.10.1/stdlib//Coq.Init.Datatypes}{\coqdocconstructor{false}} = \coqexternalref{true}{http://coq.inria.fr/distrib/V8.10.1/stdlib//Coq.Init.Datatypes}{\coqdocconstructor{true}} and \coqref{equations intro.neg}{\coqdocdefinition{neg}} \coqexternalref{true}{http://coq.inria.fr/distrib/V8.10.1/stdlib//Coq.Init.Datatypes}{\coqdocconstructor{true}} = \coqexternalref{false}{http://coq.inria.fr/distrib/V8.10.1/stdlib//Coq.Init.Datatypes}{\coqdocconstructor{false}}.
   These correspond exactly to the rewriting lemmas generated for \coqref{equations intro.neg}{\coqdocdefinition{neg}}.


   In the following sections we will show how these ideas generalize to more complex 
   types and definitions involving dependencies, overlapping clauses and recursion.


\section{Building up}




\subsection{Polymorphism}




   Coq's inductive types can be parameterized by types, giving polymorphic datatypes.
   For example the list datatype is defined as:
   \begin{coqdoccode}
\coqdocemptyline
\coqdocnoindent
\coqdockw{Inductive} \coqdef{equations intro.list}{list}{\coqdocinductive{list}} \{\coqdocvar{A}\} : \coqdockw{Type} := \coqdef{equations intro.nil}{nil}{\coqdocconstructor{nil}} : \coqref{equations intro.list}{\coqdocinductive{list}} \ensuremath{|} \coqdef{equations intro.cons}{cons}{\coqdocconstructor{cons}} : \coqdocvariable{A} \coqexternalref{::type scope:x '->' x}{http://coq.inria.fr/distrib/V8.10.1/stdlib//Coq.Init.Logic}{\coqdocnotation{\ensuremath{\rightarrow}}} \coqref{equations intro.list}{\coqdocinductive{list}} \coqexternalref{::type scope:x '->' x}{http://coq.inria.fr/distrib/V8.10.1/stdlib//Coq.Init.Logic}{\coqdocnotation{\ensuremath{\rightarrow}}} \coqref{equations intro.list}{\coqdocinductive{list}}.\coqdoceol
\coqdocemptyline
\coqdocnoindent
\coqdocvar{Arguments} \coqref{equations intro.list}{\coqdocinductive{list}} : \coqdoctac{clear} \coqdocvar{implicits}.\coqdoceol
\coqdocnoindent
\coqdockw{Notation} \coqdef{equations intro.:::x '::' x}{"}{"}x :: l" := (\coqref{equations intro.cons}{\coqdocconstructor{cons}} \coqdocvar{x} \coqdocvar{l}).\coqdoceol
\coqdocemptyline
\end{coqdoccode}
No special support for polymorphism is needed, as type arguments are treated 
   like regular arguments in dependent type theories. Note however that one cannot
   match on type arguments, there is no intensional type analysis.
   We can write the polymorphic \coqref{equations intro.tail}{\coqdocdefinition{tail}} function as follows:
\begin{coqdoccode}
\coqdocemptyline
\coqdocnoindent
\coqdockw{Equations} \coqdef{equations intro.tail}{tail}{\coqdocdefinition{tail}} \{\coqdocvar{A}\} (\coqdocvar{l} : \coqref{equations intro.list}{\coqdocinductive{list}} \coqdocvariable{A}) : \coqref{equations intro.list}{\coqdocinductive{list}} \coqdocvar{A} :=\coqdoceol
\coqdocnoindent
\coqref{equations intro.tail}{\coqdocdefinition{tail}} \coqref{equations intro.nil}{\coqdocconstructor{nil}} := \coqref{equations intro.nil}{\coqdocconstructor{nil}} ;\coqdoceol
\coqdocnoindent
\coqref{equations intro.tail}{\coqdocdefinition{tail}} (\coqref{equations intro.cons}{\coqdocconstructor{cons}} \coqdocvariable{a} \coqdocvariable{v}) := \coqdocvar{v}.\coqdoceol
\coqdocemptyline
\end{coqdoccode}
Note that the argument \{\coqdocvariable{A}\} is declared implicit and must hence be
 omitted in the defining clauses. In each of the branches it is named
 \coqdocvariable{A}. To specify it explicitely one can use the syntax (\coqdocvariable{A}:=\coqdocvariable{B}),
 renaming that implicit argument to \coqdocvariable{B} in this particular case 

\subsection{Recursive inductive types}




   Of course with inductive types comes recursion. Coq accepts a subset
   of the structurally recursive definitions by default (it is
   incomplete due to its syntactic nature). We will use this as a first
   step towards a more robust treatment of recursion via well-founded
   relations. A classical example is list concatenation: \begin{coqdoccode}
\coqdocemptyline
\coqdocnoindent
\coqdockw{Equations} \coqdef{equations intro.app}{app}{\coqdocdefinition{app}} \{\coqdocvar{A}\} (\coqdocvar{l} \coqdocvar{l'} : \coqref{equations intro.list}{\coqdocinductive{list}} \coqdocvariable{A}) : \coqref{equations intro.list}{\coqdocinductive{list}} \coqdocvar{A} :=\coqdoceol
\coqdocnoindent
\coqref{equations intro.app}{\coqdocdefinition{app}} \coqref{equations intro.nil}{\coqdocconstructor{nil}} \coqdocvariable{l'} := \coqdocvar{l'} ;\coqdoceol
\coqdocnoindent
\coqref{equations intro.app}{\coqdocdefinition{app}} (\coqref{equations intro.cons}{\coqdocconstructor{cons}} \coqdocvariable{a} \coqdocvariable{l}) \coqdocvariable{l'} := \coqref{equations intro.cons}{\coqdocconstructor{cons}} \coqdocvar{a} (\coqref{equations intro.app}{\coqdocdefinition{app}} \coqdocvar{l} \coqdocvar{l'}).\coqdoceol
\coqdocemptyline
\end{coqdoccode}
Recursive definitions like \coqref{equations intro.app}{\coqdocdefinition{app}} can be unfolded easily so proving the 
   equations as rewrite rules is direct. The induction principle associated 
   to this definition is more interesting however. We can derive from it the 
   following \textit{elimination} principle for calls to \coqref{equations intro.app}{\coqdocdefinition{app}}: \coqdoceol
\coqdocemptyline
\coqdocindent{1.50em}
\coqdocvar{app\_elim} :\coqdoceol
\coqdocindent{1.50em}
\coqdockw{\ensuremath{\forall}} \coqdocvariable{P} : \coqdockw{\ensuremath{\forall}} (\coqdocvariable{A} : \coqdockw{Type}) (\coqdocvariable{l} \coqdocvariable{l'} : \coqref{equations intro.list}{\coqdocinductive{list}} \coqdocvariable{A}), \coqref{equations intro.list}{\coqdocinductive{list}} \coqdocvariable{A} \ensuremath{\rightarrow} \coqdockw{Prop},\coqdoceol
\coqdocindent{1.50em}
(\coqdockw{\ensuremath{\forall}} (\coqdocvariable{A} : \coqdockw{Type}) (\coqdocvariable{l'} : \coqref{equations intro.list}{\coqdocinductive{list}} \coqdocvariable{A}), \coqdocvariable{P} \coqdocvariable{A} \coqref{equations intro.nil}{\coqdocconstructor{nil}} \coqdocvariable{l'} \coqdocvariable{l'}) \ensuremath{\rightarrow}\coqdoceol
\coqdocindent{1.50em}
(\coqdockw{\ensuremath{\forall}} (\coqdocvariable{A} : \coqdockw{Type}) (\coqdocvariable{a} : \coqdocvariable{A}) (\coqdocvariable{l} \coqdocvariable{l'} : \coqref{equations intro.list}{\coqdocinductive{list}} \coqdocvariable{A}),\coqdoceol
\coqdocindent{1.50em}
\coqdocvariable{P} \coqdocvariable{A} \coqdocvariable{l} \coqdocvariable{l'} (\coqref{equations intro.app}{\coqdocdefinition{app}} \coqdocvariable{l} \coqdocvariable{l'}) \ensuremath{\rightarrow} \coqdocvariable{P} \coqdocvariable{A} (\coqdocvariable{a} :: \coqdocvariable{l}) \coqdocvariable{l'} (\coqdocvariable{a} :: \coqref{equations intro.app}{\coqdocdefinition{app}} \coqdocvariable{l} \coqdocvariable{l'})) \ensuremath{\rightarrow}\coqdoceol
\coqdocindent{1.50em}
\coqdockw{\ensuremath{\forall}} (\coqdocvariable{A} : \coqdockw{Type}) (\coqdocvariable{l} \coqdocvariable{l'} : \coqref{equations intro.list}{\coqdocinductive{list}} \coqdocvariable{A}), \coqdocvariable{P} \coqdocvariable{A} \coqdocvariable{l} \coqdocvariable{l'} (\coqref{equations intro.app}{\coqdocdefinition{app}} \coqdocvariable{l} \coqdocvariable{l'}) 

\coqdocemptyline
  Using this eliminator, we can write proofs exactly following the 
  structure of the function definition, instead of redoing the splitting 
  by hand. This idea is already present in the \coqdockw{Function} package 
  \cite{Barthe:2006gp} that derives induction principles from
  function definitions.
 \begin{coqdoccode}
\coqdocemptyline
\coqdocemptyline
\end{coqdoccode}
\subsection{Moving to the left}




   The structure of real programs is richer than a simple case tree on
   the original arguments in general. In the course of a computation, we
   might want to scrutinize intermediate results (e.g. coming from
   function calls) to produce an answer. This literally means adding a
   new pattern to the left of our equations made available for further
   refinement. This concept is know as with clauses in the Agda
   \cite{norell:thesis} community and was first presented and
   implemented in the Epigram language
   \cite{DBLP:journals/jfp/McBrideM04}.


   The compilation of with clauses and its treatment for generating
   equations and the induction principle are quite involved in the
   presence of dependencies, but the basic idea is to add a new case
   analysis to the program. To compute the type of the new subprogram,
   we actually abstract the discriminee term from the expected type of
   the clause, so that the type can get refined in the subprogram. In
   the non-dependent case this does not change anything though.


   Each \coqdockw{with} node generates an auxiliary definition from the clauses
   in the curly brackets, taking the additional object as argument. The
   equation for the with node will simply be an indirection to the
   auxiliary definition and simplification will continue as usual with
   the auxiliary definition's rewrite rules.  \begin{coqdoccode}
\coqdocemptyline
\coqdocnoindent
\coqdockw{Equations} \coqdef{equations intro.filter}{filter}{\coqdocdefinition{filter}} \{\coqdocvar{A}\} (\coqdocvar{l} : \coqref{equations intro.list}{\coqdocinductive{list}} \coqdocvariable{A}) (\coqdocvar{p} : \coqdocvariable{A} \coqexternalref{::type scope:x '->' x}{http://coq.inria.fr/distrib/V8.10.1/stdlib//Coq.Init.Logic}{\coqdocnotation{\ensuremath{\rightarrow}}} \coqexternalref{bool}{http://coq.inria.fr/distrib/V8.10.1/stdlib//Coq.Init.Datatypes}{\coqdocinductive{bool}}) : \coqref{equations intro.list}{\coqdocinductive{list}} \coqdocvar{A} :=\coqdoceol
\coqdocnoindent
\coqref{equations intro.filter}{\coqdocdefinition{filter}} \coqref{equations intro.nil}{\coqdocconstructor{nil}} \coqdocvariable{p} := \coqref{equations intro.nil}{\coqdocconstructor{nil}} ;\coqdoceol
\coqdocnoindent
\coqref{equations intro.filter}{\coqdocdefinition{filter}} (\coqref{equations intro.cons}{\coqdocconstructor{cons}} \coqdocvariable{a} \coqdocvariable{l}) \coqdocvariable{p} \coqdockw{with} \coqdocvar{p} \coqdocvar{a} \ensuremath{\Rightarrow} \{\coqdoceol
\coqdocindent{1.00em}
\coqref{equations intro.filter}{\coqdocdefinition{filter}} (\coqref{equations intro.cons}{\coqdocconstructor{cons}} \coqdocvariable{a} \coqdocvariable{l}) \coqdocvariable{p} \coqexternalref{true}{http://coq.inria.fr/distrib/V8.10.1/stdlib//Coq.Init.Datatypes}{\coqdocconstructor{true}} := \coqdocvar{a} \coqref{equations intro.:::x '::' x}{\coqdocnotation{::}} \coqref{equations intro.filter}{\coqdocdefinition{filter}} \coqdocvar{l} \coqdocvar{p} ;\coqdoceol
\coqdocindent{1.00em}
\coqref{equations intro.filter}{\coqdocdefinition{filter}} (\coqref{equations intro.cons}{\coqdocconstructor{cons}} \coqdocvariable{a} \coqdocvariable{l}) \coqdocvariable{p} \coqexternalref{false}{http://coq.inria.fr/distrib/V8.10.1/stdlib//Coq.Init.Datatypes}{\coqdocconstructor{false}} := \coqref{equations intro.filter}{\coqdocdefinition{filter}} \coqdocvar{l} \coqdocvar{p} \}.\coqdoceol
\coqdocemptyline
\end{coqdoccode}
By default, equations makes definitions opaque after definition,
    to avoid spurious unfoldings, but this can be reverted on a case by case
    basis, or using the global \coqdockw{Set} \coqdockw{Equations} \coqdockw{Transparent} option. \begin{coqdoccode}
\coqdocnoindent
\coqdockw{Global Transparent} \coqref{equations intro.filter}{\coqdocdefinition{filter}}.\coqdoceol
\coqdocemptyline
\end{coqdoccode}
A common use of with clauses is to scrutinize recursive results like the following: \begin{coqdoccode}
\coqdocemptyline
\coqdocnoindent
\coqdockw{Equations} \coqdef{equations intro.unzip}{unzip}{\coqdocdefinition{unzip}} \{\coqdocvar{A} \coqdocvar{B}\} (\coqdocvar{l} : \coqref{equations intro.list}{\coqdocinductive{list}} (\coqdocvariable{A} \coqexternalref{::type scope:x '*' x}{http://coq.inria.fr/distrib/V8.10.1/stdlib//Coq.Init.Datatypes}{\coqdocnotation{\ensuremath{\times}}} \coqdocvariable{B})) : \coqref{equations intro.list}{\coqdocinductive{list}} \coqdocvar{A} \coqexternalref{::type scope:x '*' x}{http://coq.inria.fr/distrib/V8.10.1/stdlib//Coq.Init.Datatypes}{\coqdocnotation{\ensuremath{\times}}} \coqref{equations intro.list}{\coqdocinductive{list}} \coqdocvar{B} :=\coqdoceol
\coqdocnoindent
\coqref{equations intro.unzip}{\coqdocdefinition{unzip}} \coqref{equations intro.nil}{\coqdocconstructor{nil}} := \coqexternalref{::core scope:'(' x ',' x ',' '..' ',' x ')'}{http://coq.inria.fr/distrib/V8.10.1/stdlib//Coq.Init.Datatypes}{\coqdocnotation{(}}\coqref{equations intro.nil}{\coqdocconstructor{nil}}\coqexternalref{::core scope:'(' x ',' x ',' '..' ',' x ')'}{http://coq.inria.fr/distrib/V8.10.1/stdlib//Coq.Init.Datatypes}{\coqdocnotation{,}} \coqref{equations intro.nil}{\coqdocconstructor{nil}}\coqexternalref{::core scope:'(' x ',' x ',' '..' ',' x ')'}{http://coq.inria.fr/distrib/V8.10.1/stdlib//Coq.Init.Datatypes}{\coqdocnotation{)}} ;\coqdoceol
\coqdocnoindent
\coqref{equations intro.unzip}{\coqdocdefinition{unzip}} (\coqref{equations intro.cons}{\coqdocconstructor{cons}} \coqdocvariable{p} \coqdocvariable{l}) \coqdockw{with} \coqref{equations intro.unzip}{\coqdocdefinition{unzip}} \coqdocvar{l} \ensuremath{\Rightarrow} \{\coqdoceol
\coqdocindent{1.00em}
\coqref{equations intro.unzip}{\coqdocdefinition{unzip}} (\coqref{equations intro.cons}{\coqdocconstructor{cons}} \coqref{equations intro.cons}{\coqdocconstructor{(}}\coqexternalref{pair}{http://coq.inria.fr/distrib/V8.10.1/stdlib//Coq.Init.Datatypes}{\coqdocconstructor{pair}} \coqdocvariable{a} \coqdocvariable{b}\coqref{equations intro.cons}{\coqdocconstructor{)}} \coqdocvariable{l}) (\coqexternalref{pair}{http://coq.inria.fr/distrib/V8.10.1/stdlib//Coq.Init.Datatypes}{\coqdocconstructor{pair}} \coqdocvariable{la} \coqdocvariable{lb}) := \coqexternalref{::core scope:'(' x ',' x ',' '..' ',' x ')'}{http://coq.inria.fr/distrib/V8.10.1/stdlib//Coq.Init.Datatypes}{\coqdocnotation{(}}\coqdocvar{a} \coqref{equations intro.:::x '::' x}{\coqdocnotation{::}} \coqdocvar{la}\coqexternalref{::core scope:'(' x ',' x ',' '..' ',' x ')'}{http://coq.inria.fr/distrib/V8.10.1/stdlib//Coq.Init.Datatypes}{\coqdocnotation{,}} \coqdocvar{b} \coqref{equations intro.:::x '::' x}{\coqdocnotation{::}} \coqdocvar{lb}\coqexternalref{::core scope:'(' x ',' x ',' '..' ',' x ')'}{http://coq.inria.fr/distrib/V8.10.1/stdlib//Coq.Init.Datatypes}{\coqdocnotation{)}} \}.\coqdoceol
\coqdocemptyline
\end{coqdoccode}
The real power of with however comes when it is used with dependent types. 

\section{Dependent types}




   Coq supports writing dependent functions, in other words, it gives the ability to
   make the results \textit{type} depend on actual \textit{values}, like the arguments of the function.
   A simple example is given below of a function which decides the equality of two 
   natural numbers, returning a sum type carrying proofs of the equality or disequality 
   of the arguments. The sum type \{ \coqdocvariable{A} \} + \{ \coqdocvariable{B} \} is a constructive variant of disjunction 
   that can be used in programs to give at the same time a boolean algorithmic information 
   (are we in branch \coqdocvariable{A} or \coqdocvariable{B}) and a \textit{logical} information (a proof witness of \coqdocvariable{A} or \coqdocvariable{B}).
   Hence its constructors \coqdoctac{left} and \coqdoctac{right} take proofs as arguments. The \coqexternalref{eq refl}{http://coq.inria.fr/distrib/V8.10.1/stdlib//Coq.Init.Logic}{\coqdocconstructor{eq\_refl}} proof 
   term is the single proof of \coqdocvariable{x} = \coqdocvariable{x} (the \coqdocvariable{x} is generaly infered automatically).
\begin{coqdoccode}
\coqdocemptyline
\coqdocnoindent
\coqdockw{Equations} \coqdef{equations intro.equal}{equal}{\coqdocdefinition{equal}} (\coqdocvar{n} \coqdocvar{m} : \coqexternalref{nat}{http://coq.inria.fr/distrib/V8.10.1/stdlib//Coq.Init.Datatypes}{\coqdocinductive{nat}}) : \coqexternalref{::type scope:'x7B' x 'x7D' '+' 'x7B' x 'x7D'}{http://coq.inria.fr/distrib/V8.10.1/stdlib//Coq.Init.Specif}{\coqdocnotation{\{}} \coqdocvar{n} \coqexternalref{::type scope:x '=' x}{http://coq.inria.fr/distrib/V8.10.1/stdlib//Coq.Init.Logic}{\coqdocnotation{=}} \coqdocvar{m} \coqexternalref{::type scope:'x7B' x 'x7D' '+' 'x7B' x 'x7D'}{http://coq.inria.fr/distrib/V8.10.1/stdlib//Coq.Init.Specif}{\coqdocnotation{\}}} \coqexternalref{::type scope:'x7B' x 'x7D' '+' 'x7B' x 'x7D'}{http://coq.inria.fr/distrib/V8.10.1/stdlib//Coq.Init.Specif}{\coqdocnotation{+}} \coqexternalref{::type scope:'x7B' x 'x7D' '+' 'x7B' x 'x7D'}{http://coq.inria.fr/distrib/V8.10.1/stdlib//Coq.Init.Specif}{\coqdocnotation{\{}} \coqdocvar{n} \coqexternalref{::type scope:x '<>' x}{http://coq.inria.fr/distrib/V8.10.1/stdlib//Coq.Init.Logic}{\coqdocnotation{\ensuremath{\not=}}} \coqdocvar{m} \coqexternalref{::type scope:'x7B' x 'x7D' '+' 'x7B' x 'x7D'}{http://coq.inria.fr/distrib/V8.10.1/stdlib//Coq.Init.Specif}{\coqdocnotation{\}}} :=\coqdoceol
\coqdocnoindent
\coqref{equations intro.equal}{\coqdocdefinition{equal}} \coqexternalref{O}{http://coq.inria.fr/distrib/V8.10.1/stdlib//Coq.Init.Datatypes}{\coqdocconstructor{O}} \coqexternalref{O}{http://coq.inria.fr/distrib/V8.10.1/stdlib//Coq.Init.Datatypes}{\coqdocconstructor{O}} := \coqexternalref{left}{http://coq.inria.fr/distrib/V8.10.1/stdlib//Coq.Init.Specif}{\coqdocconstructor{left}} \coqexternalref{eq refl}{http://coq.inria.fr/distrib/V8.10.1/stdlib//Coq.Init.Logic}{\coqdocconstructor{eq\_refl}} ;\coqdoceol
\coqdocnoindent
\coqref{equations intro.equal}{\coqdocdefinition{equal}} (\coqexternalref{S}{http://coq.inria.fr/distrib/V8.10.1/stdlib//Coq.Init.Datatypes}{\coqdocconstructor{S}} \coqdocvariable{n}) (\coqexternalref{S}{http://coq.inria.fr/distrib/V8.10.1/stdlib//Coq.Init.Datatypes}{\coqdocconstructor{S}} \coqdocvariable{m}) \coqdockw{with} \coqref{equations intro.equal}{\coqdocdefinition{equal}} \coqdocvar{n} \coqdocvar{m} := \{\coqdoceol
\coqdocindent{1.00em}
\coqref{equations intro.equal}{\coqdocdefinition{equal}} (\coqexternalref{S}{http://coq.inria.fr/distrib/V8.10.1/stdlib//Coq.Init.Datatypes}{\coqdocconstructor{S}} \coqdocvariable{n}) (\coqexternalref{S}{http://coq.inria.fr/distrib/V8.10.1/stdlib//Coq.Init.Datatypes}{\coqdocconstructor{S}} \coqdocnotation{?(}\coqdocvariable{n}\coqdocnotation{)}) (\coqexternalref{left}{http://coq.inria.fr/distrib/V8.10.1/stdlib//Coq.Init.Specif}{\coqdocconstructor{left}} \coqexternalref{eq refl}{http://coq.inria.fr/distrib/V8.10.1/stdlib//Coq.Init.Logic}{\coqdocconstructor{eq\_refl}}) := \coqexternalref{left}{http://coq.inria.fr/distrib/V8.10.1/stdlib//Coq.Init.Specif}{\coqdocconstructor{left}} \coqexternalref{eq refl}{http://coq.inria.fr/distrib/V8.10.1/stdlib//Coq.Init.Logic}{\coqdocconstructor{eq\_refl}} ;\coqdoceol
\coqdocindent{1.00em}
\coqref{equations intro.equal}{\coqdocdefinition{equal}} (\coqexternalref{S}{http://coq.inria.fr/distrib/V8.10.1/stdlib//Coq.Init.Datatypes}{\coqdocconstructor{S}} \coqdocvariable{n}) (\coqexternalref{S}{http://coq.inria.fr/distrib/V8.10.1/stdlib//Coq.Init.Datatypes}{\coqdocconstructor{S}} \coqdocvariable{m}) (\coqexternalref{right}{http://coq.inria.fr/distrib/V8.10.1/stdlib//Coq.Init.Specif}{\coqdocconstructor{right}} \coqdocvariable{p}) := \coqexternalref{right}{http://coq.inria.fr/distrib/V8.10.1/stdlib//Coq.Init.Specif}{\coqdocconstructor{right}} \coqdocvar{\_} \} ;\coqdoceol
\coqdocnoindent
\coqref{equations intro.equal}{\coqdocdefinition{equal}} \coqdocvariable{x} \coqdocvariable{y} := \coqexternalref{right}{http://coq.inria.fr/distrib/V8.10.1/stdlib//Coq.Init.Specif}{\coqdocconstructor{right}} \coqdocvar{\_}.\coqdoceol
\coqdocemptyline
\end{coqdoccode}
Of particular interest here is the inner program refining the recursive result.
   As \coqref{equations intro.equal}{\coqdocdefinition{equal}} \coqdocvariable{n} \coqdocvariable{m} is of type \{ \coqdocvariable{n} = \coqdocvariable{m} \} + \{ \coqdocvariable{n} \ensuremath{\not=} \coqdocvariable{m} \} we have two cases to consider:



\begin{itemize}
\item  Either we are in the \coqdoctac{left} \coqdocvariable{p} case, and we know that \coqdocvariable{p} is a proof of \coqdocvariable{n} = \coqdocvariable{m},
     in which case we can do a nested match on \coqdocvariable{p}. The result of matching this equality
     proof is to unify \coqdocvariable{n} and \coqdocvariable{m}, hence the left hand side patterns become \coqexternalref{S}{http://coq.inria.fr/distrib/V8.10.1/stdlib//Coq.Init.Datatypes}{\coqdocconstructor{S}} \coqdocvariable{n} and
     \coqexternalref{S}{http://coq.inria.fr/distrib/V8.10.1/stdlib//Coq.Init.Datatypes}{\coqdocconstructor{S}} ?(\coqdocvariable{n}) and the return type of this branch is refined to \{ \coqdocvariable{n} = \coqdocvariable{n} \} + \{ \coqdocvariable{n} \ensuremath{\not=} \coqdocvariable{n} \}.
     We can easily provide a proof for the left case. 



\item  In the right case, we mark the proof unfilled with an underscore. This will
     generate an obligation for the hole, that can be filled automatically by a 
     predefined tactic or interactively by the user in proof mode (this uses the
     same obligation mechanism as the Program extension
     \cite{sozeau.Coq/FingerTrees/article}). In this case the automatic tactic 
     is able to derive by itself that \coqdocvariable{n} \ensuremath{\not=} \coqdocvariable{m} \ensuremath{\rightarrow} \coqexternalref{S}{http://coq.inria.fr/distrib/V8.10.1/stdlib//Coq.Init.Datatypes}{\coqdocconstructor{S}} \coqdocvariable{n} \ensuremath{\not=} \coqexternalref{S}{http://coq.inria.fr/distrib/V8.10.1/stdlib//Coq.Init.Datatypes}{\coqdocconstructor{S}} \coqdocvariable{m}.

\end{itemize}


   Dependent types are also useful to turn partial functions into total functions by
   restricting their domain. Typically, we can force the list passed to \coqref{equations intro.head}{\coqdocdefinition{head}} 
   to be non-empty using the specification:
\begin{coqdoccode}
\coqdocemptyline
\coqdocnoindent
\coqdockw{Equations} \coqdef{equations intro.head}{head}{\coqdocdefinition{head}} \{\coqdocvar{A}\} (\coqdocvar{l} : \coqref{equations intro.list}{\coqdocinductive{list}} \coqdocvariable{A}) (\coqdocvar{pf} : \coqdocvariable{l} \coqexternalref{::type scope:x '<>' x}{http://coq.inria.fr/distrib/V8.10.1/stdlib//Coq.Init.Logic}{\coqdocnotation{\ensuremath{\not=}}} \coqref{equations intro.nil}{\coqdocconstructor{nil}}) : \coqdocvar{A} :=\coqdoceol
\coqdocnoindent
\coqref{equations intro.head}{\coqdocdefinition{head}} \coqref{equations intro.nil}{\coqdocconstructor{nil}} \coqdocvariable{pf} \coqdockw{with} \coqdocvar{pf} \coqexternalref{eq refl}{http://coq.inria.fr/distrib/V8.10.1/stdlib//Coq.Init.Logic}{\coqdocconstructor{eq\_refl}} := \{ \ensuremath{|} \coqdocvariable{x} :=! \coqdocvar{x} \};\coqdoceol
\coqdocnoindent
\coqref{equations intro.head}{\coqdocdefinition{head}} (\coqref{equations intro.cons}{\coqdocconstructor{cons}} \coqdocvariable{a} \coqdocvariable{v}) \coqdocvar{\_} := \coqdocvar{a}.\coqdoceol
\coqdocemptyline
\end{coqdoccode}
We decompose the list and are faced with two cases:



\begin{itemize}
\item  In the first case, the list is empty, hence the proof \coqdocvariable{pf} of type
     \coqref{equations intro.nil}{\coqdocconstructor{nil}} \ensuremath{\not=} \coqref{equations intro.nil}{\coqdocconstructor{nil}} allows us to derive a contradiction by applying it to
     reflexivity.  We make use of another category of right-hand sides,
     which we call \textit{empty} nodes to inform the compiler that a
     contradiction is derivable in this case.  In general we cannot
     expect the compiler to find by himself that the context contains a
     contradiction, as it is undecidable
     (\cite{DBLP:conf/plpv/Oury07,DBLP:conf/birthday/GoguenMM06}).


     However, in this case, one could also write an empty set of clauses
     for the \coqdockw{with} subprogram, as Equations applies a heuristic in case
     of an empty set of clause: it tries to split each of the variables
     in the context to find an empty type.



\item  In the second case, we simply return the head of the list,
     disregarding the proof.  
\end{itemize}


\subsection{Inductive families}




   The next step is to make constraints such as non-emptiness part of the 
   datatype itself. This capability is provided through inductive families in
   Coq \cite{paulin93tlca}, which are a similar concept to the generalization 
   of algebraic datatypes to GADTs in functional languages like Haskell 
   \cite{GADTcomplete}. Families provide a way to associate to each constructor 
   a different type, making it possible to give specific information about a value 
   in its type. 


\subsubsection{Equality}


   The alma mater of inductive families is the propositional equality 
   \coqexternalref{eq}{http://coq.inria.fr/distrib/V8.10.1/stdlib//Coq.Init.Logic}{\coqdocinductive{eq}} defined as: \coqdoceol
\coqdocemptyline
\coqdocnoindent
\coqdockw{Inductive} \coqexternalref{eq}{http://coq.inria.fr/distrib/V8.10.1/stdlib//Coq.Init.Logic}{\coqdocinductive{eq}} (\coqdocvariable{A} : \coqdockw{Type}) (\coqdocvariable{x} : \coqdocvariable{A}) : \coqdocvariable{A} \ensuremath{\rightarrow} \coqdockw{Prop} := \coqdoceol
\coqdocindent{0.50em}
\coqexternalref{eq refl}{http://coq.inria.fr/distrib/V8.10.1/stdlib//Coq.Init.Logic}{\coqdocconstructor{eq\_refl}} : \coqexternalref{eq}{http://coq.inria.fr/distrib/V8.10.1/stdlib//Coq.Init.Logic}{\coqdocinductive{eq}} \coqdocvariable{A} \coqdocvariable{x} \coqdocvariable{x}. 

\coqdocemptyline


   Equality is a polymorphic relation on \coqdocvariable{A}. (The \coqdockw{Prop} sort (or kind) categorizes
   propositions, while the \coqdockw{Set} sort, equivalent to $\star$ in Haskell categorizes 
   computational types.) Equality is \textit{parameterized} by a value \coqdocvariable{x} of type \coqdocvariable{A} and 
   \textit{indexed} by another value of type \coqdocvariable{A}. Its single constructor states that 
   equality is reflexive, so the only way to build an object of \coqexternalref{eq}{http://coq.inria.fr/distrib/V8.10.1/stdlib//Coq.Init.Logic}{\coqdocinductive{eq}} \coqdocvariable{x} \coqdocvariable{y} is if 
   \coqdocvariable{x} \~{}= \coqdocvariable{y}, that is if \coqdocvariable{x} is definitionaly equal to \coqdocvariable{y}. 


   Now what is the elimination principle associated to this inductive family?
   It is the good old Leibniz substitution principle: \coqdoceol
\coqdocemptyline
\coqdocnoindent
\coqdockw{\ensuremath{\forall}} (\coqdocvariable{A} : \coqdockw{Type}) (\coqdocvariable{x} : \coqdocvariable{A}) (\coqdocvariable{P} : \coqdocvariable{A} \ensuremath{\rightarrow} \coqdockw{Type}), \coqdocvariable{P} \coqdocvariable{x} \ensuremath{\rightarrow} \coqdockw{\ensuremath{\forall}} \coqdocvariable{y} : \coqdocvariable{A}, \coqdocvariable{x} = \coqdocvariable{y} \ensuremath{\rightarrow} \coqdocvariable{P} \coqdocvariable{y} 

\coqdocemptyline


   Provided a proof that \coqdocvariable{x} = \coqdocvariable{y}, we can create on object of type \coqdocvariable{P} \coqdocvariable{y} from an 
   existing object of type \coqdocvariable{P} \coqdocvariable{x}. This substitution principle is enough to show
   that equality is symmetric and transitive. For example we can use 
   pattern-matching on equality proofs to show:
 \begin{coqdoccode}
\coqdocemptyline
\coqdocnoindent
\coqdockw{Equations} \coqdef{equations intro.eqt}{eqt}{\coqdocdefinition{eqt}} \{\coqdocvar{A}\} (\coqdocvar{x} \coqdocvar{y} \coqdocvar{z} : \coqdocvariable{A}) (\coqdocvar{p} : \coqdocvariable{x} \coqexternalref{::type scope:x '=' x}{http://coq.inria.fr/distrib/V8.10.1/stdlib//Coq.Init.Logic}{\coqdocnotation{=}} \coqdocvariable{y}) (\coqdocvar{q} : \coqdocvariable{y} \coqexternalref{::type scope:x '=' x}{http://coq.inria.fr/distrib/V8.10.1/stdlib//Coq.Init.Logic}{\coqdocnotation{=}} \coqdocvariable{z}) : \coqdocvar{x} \coqexternalref{::type scope:x '=' x}{http://coq.inria.fr/distrib/V8.10.1/stdlib//Coq.Init.Logic}{\coqdocnotation{=}} \coqdocvar{z} :=\coqdoceol
\coqdocnoindent
\coqref{equations intro.eqt}{\coqdocdefinition{eqt}} \coqdocvariable{x} \coqdocnotation{?(}\coqdocvariable{x}\coqdocnotation{)} \coqdocnotation{?(}\coqdocvariable{x}\coqdocnotation{)} \coqexternalref{eq refl}{http://coq.inria.fr/distrib/V8.10.1/stdlib//Coq.Init.Logic}{\coqdocconstructor{eq\_refl}} \coqexternalref{eq refl}{http://coq.inria.fr/distrib/V8.10.1/stdlib//Coq.Init.Logic}{\coqdocconstructor{eq\_refl}} := \coqexternalref{eq refl}{http://coq.inria.fr/distrib/V8.10.1/stdlib//Coq.Init.Logic}{\coqdocconstructor{eq\_refl}}.\coqdoceol
\coqdocemptyline
\end{coqdoccode}
Let us explain the meaning of the non-linear patterns here that we
   slipped through in the \coqref{equations intro.equal}{\coqdocdefinition{equal}} example. By pattern-matching on the
   equalities, we have unified \coqdocvariable{x}, \coqdocvariable{y} and \coqdocvariable{z}, hence we determined the
   \textit{values} of the patterns for the variables to be \coqdocvariable{x}. The ?(\coqdocvariable{x})
   notation is essentially denoting that the pattern is not a candidate
   for refinement, as it is determined by another pattern. This
   particular patterns are called ``inaccessible''. When they are variables
   the inaccessibility annotation is optional.


\subsubsection{Indexed datatypes}




   Functions on \coqref{equations intro.vector}{\coqdocinductive{vector}}s provide more stricking examples of this
   situation.  The \coqref{equations intro.vector}{\coqdocinductive{vector}} family is indexed by a natural number
   representing the size of the vector: [ \coqdockw{Inductive} \coqref{equations intro.vector}{\coqdocinductive{vector}} (\coqdocvariable{A} : \coqdockw{Type}) :
   \coqexternalref{nat}{http://coq.inria.fr/distrib/V8.10.1/stdlib//Coq.Init.Datatypes}{\coqdocinductive{nat}} \ensuremath{\rightarrow} \coqdockw{Type} := \ensuremath{|} \coqref{equations intro.Vnil}{\coqdocconstructor{Vnil}} : \coqref{equations intro.vector}{\coqdocinductive{vector}} \coqdocvariable{A} \coqexternalref{O}{http://coq.inria.fr/distrib/V8.10.1/stdlib//Coq.Init.Datatypes}{\coqdocconstructor{O}} \ensuremath{|} \coqref{equations intro.Vcons}{\coqdocconstructor{Vcons}} : \coqdocvariable{A} \ensuremath{\rightarrow} \coqdockw{\ensuremath{\forall}} \coqdocvariable{n} : \coqexternalref{nat}{http://coq.inria.fr/distrib/V8.10.1/stdlib//Coq.Init.Datatypes}{\coqdocinductive{nat}},
   \coqref{equations intro.vector}{\coqdocinductive{vector}} \coqdocvariable{A} \coqdocvariable{n} \ensuremath{\rightarrow} \coqref{equations intro.vector}{\coqdocinductive{vector}} \coqdocvariable{A} (\coqexternalref{S}{http://coq.inria.fr/distrib/V8.10.1/stdlib//Coq.Init.Datatypes}{\coqdocconstructor{S}} \coqdocvariable{n}) ]


   The empty vector \coqref{equations intro.Vnil}{\coqdocconstructor{Vnil}} has size \coqexternalref{O}{http://coq.inria.fr/distrib/V8.10.1/stdlib//Coq.Init.Datatypes}{\coqdocconstructor{O}} while the cons operation
   increments the size by one. Now let us define the usual map on
   vectors: \begin{coqdoccode}
\coqdocemptyline
\coqdocnoindent
\coqdockw{Inductive} \coqdef{equations intro.vector}{vector}{\coqdocinductive{vector}} (\coqdocvar{A} : \coqdockw{Type}) : \coqexternalref{nat}{http://coq.inria.fr/distrib/V8.10.1/stdlib//Coq.Init.Datatypes}{\coqdocinductive{nat}} \coqexternalref{::type scope:x '->' x}{http://coq.inria.fr/distrib/V8.10.1/stdlib//Coq.Init.Logic}{\coqdocnotation{\ensuremath{\rightarrow}}} \coqdockw{Type} :=\coqdoceol
\coqdocnoindent
\ensuremath{|} \coqdef{equations intro.Vnil}{Vnil}{\coqdocconstructor{Vnil}} : \coqref{equations intro.vector}{\coqdocinductive{vector}} \coqdocvariable{A} 0\coqdoceol
\coqdocnoindent
\ensuremath{|} \coqdef{equations intro.Vcons}{Vcons}{\coqdocconstructor{Vcons}} \{\coqdocvar{n}\} : \coqdocvariable{A} \coqexternalref{::type scope:x '->' x}{http://coq.inria.fr/distrib/V8.10.1/stdlib//Coq.Init.Logic}{\coqdocnotation{\ensuremath{\rightarrow}}} \coqref{equations intro.vector}{\coqdocinductive{vector}} \coqdocvariable{A} \coqdocvariable{n} \coqexternalref{::type scope:x '->' x}{http://coq.inria.fr/distrib/V8.10.1/stdlib//Coq.Init.Logic}{\coqdocnotation{\ensuremath{\rightarrow}}} \coqref{equations intro.vector}{\coqdocinductive{vector}} \coqdocvariable{A} (\coqexternalref{S}{http://coq.inria.fr/distrib/V8.10.1/stdlib//Coq.Init.Datatypes}{\coqdocconstructor{S}} \coqdocvariable{n}).\coqdoceol
\coqdocnoindent
\coqdocvar{Arguments} \coqref{equations intro.Vnil}{\coqdocconstructor{Vnil}} \{\coqdocvar{A}\}.\coqdoceol
\coqdocnoindent
\coqdocvar{Arguments} \coqref{equations intro.Vcons}{\coqdocconstructor{Vcons}} \{\coqdocvar{A} \coqdocvar{n}\}.\coqdoceol
\coqdocemptyline
\coqdocnoindent
\coqdockw{Equations} \coqdef{equations intro.vmap}{vmap}{\coqdocdefinition{vmap}} \{\coqdocvar{A} \coqdocvar{B}\} (\coqdocvar{f} : \coqdocvariable{A} \coqexternalref{::type scope:x '->' x}{http://coq.inria.fr/distrib/V8.10.1/stdlib//Coq.Init.Logic}{\coqdocnotation{\ensuremath{\rightarrow}}} \coqdocvariable{B}) \{\coqdocvar{n}\} (\coqdocvar{v} : \coqref{equations intro.vector}{\coqdocinductive{vector}} \coqdocvariable{A} \coqdocvariable{n}) :\coqdoceol
\coqdocindent{1.00em}
\coqref{equations intro.vector}{\coqdocinductive{vector}} \coqdocvar{B} \coqdocvar{n} :=\coqdoceol
\coqdocnoindent
\coqref{equations intro.vmap}{\coqdocdefinition{vmap}} \coqdocvariable{f} (\coqdocvar{n}:=\coqdocnotation{?(}0\coqdocnotation{)}) \coqref{equations intro.Vnil}{\coqdocconstructor{Vnil}} := \coqref{equations intro.Vnil}{\coqdocconstructor{Vnil}} ;\coqdoceol
\coqdocnoindent
\coqref{equations intro.vmap}{\coqdocdefinition{vmap}} \coqdocvariable{f} (\coqref{equations intro.Vcons}{\coqdocconstructor{Vcons}} \coqdocvariable{a} \coqdocvariable{v}) := \coqref{equations intro.Vcons}{\coqdocconstructor{Vcons}} (\coqdocvar{f} \coqdocvar{a}) (\coqref{equations intro.vmap}{\coqdocdefinition{vmap}} \coqdocvar{f} \coqdocvar{v}).\coqdoceol
\coqdocemptyline
\end{coqdoccode}
Here the value of the index representing the size of the vector 
   is directly determined by the constructor, hence in the case tree
   we have no need to eliminate \coqdocvariable{n}. This means in particular that 
   the function \coqref{equations intro.vmap}{\coqdocdefinition{vmap}} does not do any computation with \coqdocvariable{n}, and 
   the argument could be eliminated in the extracted code.
   In other words, it provides only \textit{logical} information about 
   the shape of \coqdocvariable{v} but no computational information.


   The \coqref{equations intro.vmap}{\coqdocdefinition{vmap}} function works on every member of the \coqref{equations intro.vector}{\coqdocinductive{vector}} family,
   but some functions may work only for some subfamilies, for example
   \coqref{equations intro.vtail}{\coqdocdefinition{vtail}}:
 \begin{coqdoccode}
\coqdocemptyline
\coqdocnoindent
\coqdockw{Equations} \coqdef{equations intro.vtail}{vtail}{\coqdocdefinition{vtail}} \{\coqdocvar{A} \coqdocvar{n}\} (\coqdocvar{v} : \coqref{equations intro.vector}{\coqdocinductive{vector}} \coqdocvariable{A} (\coqexternalref{S}{http://coq.inria.fr/distrib/V8.10.1/stdlib//Coq.Init.Datatypes}{\coqdocconstructor{S}} \coqdocvariable{n})) : \coqref{equations intro.vector}{\coqdocinductive{vector}} \coqdocvar{A} \coqdocvar{n} :=\coqdoceol
\coqdocnoindent
\coqref{equations intro.vtail}{\coqdocdefinition{vtail}} (\coqref{equations intro.Vcons}{\coqdocconstructor{Vcons}} \coqdocvariable{a} \coqdocvariable{v'}) := \coqdocvar{v'}.\coqdoceol
\coqdocemptyline
\end{coqdoccode}
The type of \coqdocvariable{v} ensures that \coqref{equations intro.vtail}{\coqdocdefinition{vtail}} can only be applied to 
   non-empty vectors, moreover the patterns only need to consider 
   constructors that can produce objects in the subfamily \coqref{equations intro.vector}{\coqdocinductive{vector}} \coqdocvariable{A} (\coqexternalref{S}{http://coq.inria.fr/distrib/V8.10.1/stdlib//Coq.Init.Datatypes}{\coqdocconstructor{S}} \coqdocvariable{n}),
   excluding \coqref{equations intro.Vnil}{\coqdocconstructor{Vnil}}. The pattern-matching compiler uses unification 
   with the theory of constructors to discover which cases need to 
   be considered and which are impossible. In this case the failed 
   unification of 0 and \coqexternalref{S}{http://coq.inria.fr/distrib/V8.10.1/stdlib//Coq.Init.Datatypes}{\coqdocconstructor{S}} \coqdocvariable{n} shows that the \coqref{equations intro.Vnil}{\coqdocconstructor{Vnil}} case is impossible.
   This powerful unification engine running under the hood permits to write
   concise code where all uninteresting cases are handled automatically. 

\subsection{Derived notions, No-Confusion}




    For this to work smoothlty, the package requires some derived definitions
    on each (indexed) family, which can be generated automatically using
    the generic \coqdocvar{\coqdockw{Derive}} command. Here we ask to generate the signature,
    heterogeneous no-confusion and homogeneous no-confusion principles for vectors: \begin{coqdoccode}
\coqdocemptyline
\coqdocnoindent
\coqdocvar{\coqdockw{Derive}} \coqdocvar{\coqdocind{NoConfusion}} \coqdockw{for} \coqexternalref{nat}{http://coq.inria.fr/distrib/V8.10.1/stdlib//Coq.Init.Datatypes}{\coqdocinductive{nat}}.\coqdoceol
\coqdocnoindent
\coqdocvar{\coqdockw{Derive}} \coqdocvar{\coqdocind{Signature}} \coqdocvar{\coqdocind{NoConfusion}} \coqdocvar{NoConfusionHom} \coqdockw{for} \coqref{equations intro.vector}{\coqdocinductive{vector}}.\coqdoceol
\coqdocemptyline
\end{coqdoccode}
The precise specification of these derived definitions can be found in the manual
    section (\S \ref{manual}). Signature is used to ``pack'' a value in an inductive family
    with its index, e.g. the ``total space'' of every index and value of the family. This
    can be used to derive the heterogeneous no-confusion principle for the family, which
    allows to discriminate between objects in potentially different instances/fibers of the family,
    or deduce injectivity of each constructor. The \coqdocvar{NoConfusionHom} variant derives
    the homogeneous no-confusion principle between two objects in the \textit{same} instance
    of the family, e.g. to simplify equations of the form \coqref{equations intro.Vnil}{\coqdocconstructor{Vnil}} = \coqref{equations intro.Vnil}{\coqdocconstructor{Vnil}} :> \coqref{equations intro.vector}{\coqdocinductive{vector}} \coqdocvariable{A} 0.
    This last principle can only be defined when pattern-matching on the inductive family
    does not require the \coqref{equations intro.KAxiom.WithAx.K}{\coqdocdefinition{K}} axiom and will otherwise fail.


\subsection{Unification and indexed datatypes}




   Back to our example, of course the equations and the induction principle are simplified in a
   similar way. If we encounter a call to \coqref{equations intro.vtail}{\coqdocdefinition{vtail}} in a proof, we can 
   use the following elimination principle to simplify both the call and the
   argument which will be automatically substituted by an object of the form
   \coqref{equations intro.Vcons}{\coqdocconstructor{Vcons}} \coqdocvar{\_} \coqdocvar{\_} \coqdocvar{\_}:\coqdoceol
\coqdocemptyline
\coqdocnoindent
\coqdockw{\ensuremath{\forall}} \coqdocvariable{P} : \coqdockw{\ensuremath{\forall}} (\coqdocvariable{A} : \coqdockw{Type}) (\coqdocvariable{n} : \coqexternalref{nat}{http://coq.inria.fr/distrib/V8.10.1/stdlib//Coq.Init.Datatypes}{\coqdocinductive{nat}}), \coqref{equations intro.vector}{\coqdocinductive{vector}} \coqdocvariable{A} (\coqexternalref{S}{http://coq.inria.fr/distrib/V8.10.1/stdlib//Coq.Init.Datatypes}{\coqdocconstructor{S}} \coqdocvariable{n}) \ensuremath{\rightarrow} \coqref{equations intro.vector}{\coqdocinductive{vector}} \coqdocvariable{A} \coqdocvariable{n} \ensuremath{\rightarrow} \coqdockw{Prop},\coqdoceol
\coqdocnoindent
(\coqdockw{\ensuremath{\forall}} (\coqdocvariable{A} : \coqdockw{Type}) (\coqdocvariable{n} : \coqexternalref{nat}{http://coq.inria.fr/distrib/V8.10.1/stdlib//Coq.Init.Datatypes}{\coqdocinductive{nat}}) (\coqdocvariable{a} : \coqdocvariable{A}) (\coqdocvariable{v} : \coqref{equations intro.vector}{\coqdocinductive{vector}} \coqdocvariable{A} \coqdocvariable{n}), \coqdoceol
\coqdocindent{1.00em}
\coqdocvariable{P} \coqdocvariable{A} \coqdocvariable{n} (\coqref{equations intro.Vcons}{\coqdocconstructor{Vcons}} \coqdocvariable{a} \coqdocvariable{v}) \coqdocvariable{v}) \ensuremath{\rightarrow}\coqdoceol
\coqdocnoindent
\coqdockw{\ensuremath{\forall}} (\coqdocvariable{A} : \coqdockw{Type}) (\coqdocvariable{n} : \coqexternalref{nat}{http://coq.inria.fr/distrib/V8.10.1/stdlib//Coq.Init.Datatypes}{\coqdocinductive{nat}}) (\coqdocvariable{v} : \coqref{equations intro.vector}{\coqdocinductive{vector}} \coqdocvariable{A} (\coqexternalref{S}{http://coq.inria.fr/distrib/V8.10.1/stdlib//Coq.Init.Datatypes}{\coqdocconstructor{S}} \coqdocvariable{n})), \coqdocvariable{P} \coqdocvariable{A} \coqdocvariable{n} \coqdocvariable{v} (\coqref{equations intro.vtail}{\coqdocdefinition{vtail}} \coqdocvariable{v}) 

\coqdocemptyline


   As a witness of the power of the unification, consider the following function 
   which computes the diagonal of a square matrix of size \coqdocvariable{n} \ensuremath{\times} \coqdocvariable{n}.
\begin{coqdoccode}
\coqdocemptyline
\coqdocnoindent
\coqdockw{Equations} \coqdef{equations intro.diag}{diag}{\coqdocdefinition{diag}} \{\coqdocvar{A} \coqdocvar{n}\} (\coqdocvar{v} : \coqref{equations intro.vector}{\coqdocinductive{vector}} (\coqref{equations intro.vector}{\coqdocinductive{vector}} \coqdocvariable{A} \coqdocvariable{n}) \coqdocvariable{n}) : \coqref{equations intro.vector}{\coqdocinductive{vector}} \coqdocvar{A} \coqdocvar{n} :=\coqdoceol
\coqdocnoindent
\coqref{equations intro.diag}{\coqdocdefinition{diag}} (\coqdocvar{n}:=\coqexternalref{O}{http://coq.inria.fr/distrib/V8.10.1/stdlib//Coq.Init.Datatypes}{\coqdocconstructor{O}}) \coqref{equations intro.Vnil}{\coqdocconstructor{Vnil}} := \coqref{equations intro.Vnil}{\coqdocconstructor{Vnil}} ;\coqdoceol
\coqdocnoindent
\coqref{equations intro.diag}{\coqdocdefinition{diag}} (\coqdocvar{n}:=\coqexternalref{S}{http://coq.inria.fr/distrib/V8.10.1/stdlib//Coq.Init.Datatypes}{\coqdocconstructor{S}} \coqexternalref{S}{http://coq.inria.fr/distrib/V8.10.1/stdlib//Coq.Init.Datatypes}{\coqdocconstructor{\_}}) (\coqref{equations intro.Vcons}{\coqdocconstructor{Vcons}} \coqref{equations intro.Vcons}{\coqdocconstructor{(}}\coqref{equations intro.Vcons}{\coqdocconstructor{Vcons}} \coqdocvariable{a} \coqdocvariable{v}\coqref{equations intro.Vcons}{\coqdocconstructor{)}} \coqdocvariable{v'}) :=\coqdoceol
\coqdocindent{1.00em}
\coqref{equations intro.Vcons}{\coqdocconstructor{Vcons}} \coqdocvar{a} (\coqref{equations intro.diag}{\coqdocdefinition{diag}} (\coqref{equations intro.vmap}{\coqdocdefinition{vmap}} \coqref{equations intro.vtail}{\coqdocdefinition{vtail}} \coqdocvar{v'})).\coqdoceol
\coqdocemptyline
\end{coqdoccode}
Here in the second equation, we know that the elements of the vector
   are necessarily of size \coqexternalref{S}{http://coq.inria.fr/distrib/V8.10.1/stdlib//Coq.Init.Datatypes}{\coqdocconstructor{S}} \coqdocvariable{n} too, hence we can do a nested refinement
   on the first one to find the first element of the diagonal.
  

\subsection{Recursion}




  Notice how in the \coqref{equations intro.diag}{\coqdocdefinition{diag}} example above we explicitely pattern-matched
  on the index \coqdocvariable{n}, even though the \coqref{equations intro.Vnil}{\coqdocconstructor{Vnil}} and \coqref{equations intro.Vcons}{\coqdocconstructor{Vcons}} pattern matching
  would have been enough to determine these indices. This is because the
  following definition also fails: \begin{coqdoccode}
\coqdocemptyline
\coqdocnoindent
\coqdocvar{Fail} \coqdockw{Equations} \coqdef{equations intro.diag'}{diag'}{\coqdocdefinition{diag'}} \{\coqdocvar{A} \coqdocvar{n}\} (\coqdocvar{v} : \coqref{equations intro.vector}{\coqdocinductive{vector}} (\coqref{equations intro.vector}{\coqdocinductive{vector}} \coqdocvariable{A} \coqdocvariable{n}) \coqdocvariable{n}) : \coqref{equations intro.vector}{\coqdocinductive{vector}} \coqdocvar{A} \coqdocvar{n} :=\coqdoceol
\coqdocnoindent
\coqref{equations intro.diag'}{\coqdocdefinition{diag'}} \coqref{equations intro.Vnil}{\coqdocconstructor{Vnil}} := \coqref{equations intro.Vnil}{\coqdocconstructor{Vnil}} ;\coqdoceol
\coqdocnoindent
\coqref{equations intro.diag'}{\coqdocdefinition{diag'}} (\coqref{equations intro.Vcons}{\coqdocconstructor{Vcons}} \coqref{equations intro.Vcons}{\coqdocconstructor{(}}\coqref{equations intro.Vcons}{\coqdocconstructor{Vcons}} \coqdocvariable{a} \coqdocvariable{v}\coqref{equations intro.Vcons}{\coqdocconstructor{)}} \coqdocvariable{v'}) :=\coqdoceol
\coqdocindent{1.00em}
\coqref{equations intro.Vcons}{\coqdocconstructor{Vcons}} \coqdocvar{a} (\coqref{equations intro.diag'}{\coqdocdefinition{diag'}} (\coqref{equations intro.vmap}{\coqdocdefinition{vmap}} \coqref{equations intro.vtail}{\coqdocdefinition{vtail}} \coqdocvar{v'})).\coqdoceol
\coqdocemptyline
\end{coqdoccode}
Indeed, Coq cannot guess the decreasing argument of this fixpoint
    using its limited syntactic guard criterion: \coqref{equations intro.vmap}{\coqdocdefinition{vmap}} \coqref{equations intro.vtail}{\coqdocdefinition{vtail}} \coqdocvariable{v'} cannot
    be seen to be a (large) subterm of \coqdocvariable{v'} using this criterion, even
    if it is clearly ``smaller''. In general, it can also be the case that
    the compilation algorithm introduces decorations to the proof term
    that prevent the syntactic guard check from seeing that the
    definition is structurally recursive.


    To aleviate this problem, \coqdockw{Equations} provides support for
    \textit{well-founded} recursive definitions which do not rely on syntactic
    checks.


    The simplest example of this is using the \coqexternalref{lt}{http://coq.inria.fr/distrib/V8.10.1/stdlib//Coq.Init.Peano}{\coqdocdefinition{lt}} order on natural numbers
    to define a recursive definition of identity: \begin{coqdoccode}
\coqdocemptyline
\coqdocnoindent
\coqdockw{Equations} \coqdef{equations intro.id}{id}{\coqdocdefinition{id}} (\coqdocvar{n} : \coqexternalref{nat}{http://coq.inria.fr/distrib/V8.10.1/stdlib//Coq.Init.Datatypes}{\coqdocinductive{nat}}) : \coqexternalref{nat}{http://coq.inria.fr/distrib/V8.10.1/stdlib//Coq.Init.Datatypes}{\coqdocinductive{nat}} \coqdoctac{by} \coqdockw{wf} \coqdocvar{n} \coqexternalref{lt}{http://coq.inria.fr/distrib/V8.10.1/stdlib//Coq.Init.Peano}{\coqdocdefinition{lt}} :=\coqdoceol
\coqdocindent{1.00em}
\coqref{equations intro.id}{\coqdocdefinition{id}} \coqexternalref{O}{http://coq.inria.fr/distrib/V8.10.1/stdlib//Coq.Init.Datatypes}{\coqdocconstructor{0}} := 0;\coqdoceol
\coqdocindent{1.00em}
\coqref{equations intro.id}{\coqdocdefinition{id}} (\coqexternalref{S}{http://coq.inria.fr/distrib/V8.10.1/stdlib//Coq.Init.Datatypes}{\coqdocconstructor{S}} \coqdocvariable{n'}) := \coqexternalref{S}{http://coq.inria.fr/distrib/V8.10.1/stdlib//Coq.Init.Datatypes}{\coqdocconstructor{S}} (\coqref{equations intro.id}{\coqdocdefinition{id}} \coqdocvar{n'}).\coqdoceol
\coqdocemptyline
\end{coqdoccode}
Here \coqref{equations intro.id}{\coqdocdefinition{id}} is defined by well-founded recursion on \coqexternalref{lt}{http://coq.inria.fr/distrib/V8.10.1/stdlib//Coq.Init.Peano}{\coqdocdefinition{lt}} on the (only)
    argument \coqdocvariable{n} using the \coqdoctac{by} \coqdockw{wf} annotation.  At recursive calls of
    \coqref{equations intro.id}{\coqdocdefinition{id}}, obligations are generated to show that the arguments
    effectively decrease according to this relation.  Here the proof
    that \coqdocvariable{n'} < \coqexternalref{S}{http://coq.inria.fr/distrib/V8.10.1/stdlib//Coq.Init.Datatypes}{\coqdocconstructor{S}} \coqdocvariable{n'} is discharged automatically.


  Wellfounded recursion on arbitrary dependent families is not as easy
  to use, as in general the relations on families are \textit{heterogeneous},
  as they must relate inhabitants of potentially different instances of
  the family.  \coqdockw{Equations} provides a \coqdocvar{\coqdockw{Derive}} command to generate the
  subterm relation on any such inductive family and derive the
  well-foundedness of its transitive closure. This provides
  course-of-values or so-called ``mathematical'' induction on these
  objects, reflecting the structural recursion criterion in the logic. \begin{coqdoccode}
\coqdocemptyline
\coqdocnoindent
\coqdocvar{\coqdockw{Derive}} \coqdocvar{Subterm} \coqdockw{for} \coqref{equations intro.vector}{\coqdocinductive{vector}}.\coqdoceol
\coqdocemptyline
\end{coqdoccode}
For vectors for example, the relation is defined as: \coqdoceol
\coqdocemptyline
\coqdocnoindent
\coqdockw{Inductive} \coqdocvar{vector\_direct\_subterm} (\coqdocvariable{A} : \coqdockw{Type}) :\coqdoceol
\coqdocindent{1.00em}
\coqdockw{\ensuremath{\forall}} \coqdocvariable{n} \coqdocvar{n0} : \coqexternalref{nat}{http://coq.inria.fr/distrib/V8.10.1/stdlib//Coq.Init.Datatypes}{\coqdocinductive{nat}}, \coqref{equations intro.vector}{\coqdocinductive{vector}} \coqdocvariable{A} \coqdocvariable{n} \ensuremath{\rightarrow} \coqref{equations intro.vector}{\coqdocinductive{vector}} \coqdocvariable{A} \coqdocvar{n0} \ensuremath{\rightarrow} \coqdockw{Prop} :=\coqdoceol
\coqdocindent{2.00em}
\coqdocvar{vector\_direct\_subterm\_1\_1} : \coqdockw{\ensuremath{\forall}} (\coqdocvar{h} : \coqdocvariable{A}) (\coqdocvariable{n} : \coqexternalref{nat}{http://coq.inria.fr/distrib/V8.10.1/stdlib//Coq.Init.Datatypes}{\coqdocinductive{nat}}) (\coqdocvar{H} : \coqref{equations intro.vector}{\coqdocinductive{vector}} \coqdocvariable{A} \coqdocvariable{n}),\coqdoceol
\coqdocindent{3.00em}
\coqdocvar{vector\_direct\_subterm} \coqdocvariable{A} \coqdocvariable{n} (\coqexternalref{S}{http://coq.inria.fr/distrib/V8.10.1/stdlib//Coq.Init.Datatypes}{\coqdocconstructor{S}} \coqdocvariable{n}) \coqdocvar{H} (\coqref{equations intro.Vcons}{\coqdocconstructor{Vcons}} \coqdocvar{h} \coqdocvar{H}) 

\coqdocemptyline


  That is, there is only one recursive subterm, for the subvector
  in the \coqref{equations intro.Vcons}{\coqdocconstructor{Vcons}} constructor. We also get a proof of:
 \begin{coqdoccode}
\coqdocemptyline
\coqdocnoindent
\coqdockw{Check} \coqref{equations intro.well founded vector subterm}{\coqdocinstance{well\_founded\_vector\_subterm}} : \coqdockw{\ensuremath{\forall}} \coqdocvar{A}, \coqdocclass{WellFounded} (\coqref{equations intro.vector subterm}{\coqdocdefinition{vector\_subterm}} \coqdocvariable{A}).\coqdoceol
\coqdocemptyline
\end{coqdoccode}
The relation is actually called \coqdocvar{t\_subterm} as \coqref{equations intro.vector}{\coqdocinductive{vector}} is just
    a notation for \coqdocvar{Vector.t}.
    \coqdocvar{t\_subterm} itself is the transitive closure of the relation seen as
    an homogeneous one by packing the indices of the family with the
    object itself. Once this is derived, we can use it to define
    recursive definitions on vectors that the guard condition couldn't
    handle. The signature provides a \coqdocdefinition{signature\_pack} function to pack a
    vector with its index. The well-founded relation is defined on the
    packed vector type. \begin{coqdoccode}
\coqdocemptyline
\coqdocnoindent
\coqdockw{Module} \coqdef{equations intro.UnzipVect}{UnzipVect}{\coqdocmodule{UnzipVect}}.\coqdoceol
\coqdocindent{1.00em}
\coqdockw{Context} \{\coqdocvar{A} \coqdocvar{B} : \coqdockw{Type}\}.\coqdoceol
\coqdocemptyline
\end{coqdoccode}
We can use the packed relation to do well-founded recursion on the vector.
      Note that we do a recursive call on a substerm of type \coqref{equations intro.vector}{\coqdocinductive{vector}} \coqdocvariable{A} \coqdocvariable{n} which
      must be shown smaller than a \coqref{equations intro.vector}{\coqdocinductive{vector}} \coqdocvariable{A} (\coqexternalref{S}{http://coq.inria.fr/distrib/V8.10.1/stdlib//Coq.Init.Datatypes}{\coqdocconstructor{S}} \coqdocvariable{n}). They are actually compared
      at the packed type \{ \coqdocvariable{n} : \coqexternalref{nat}{http://coq.inria.fr/distrib/V8.10.1/stdlib//Coq.Init.Datatypes}{\coqdocinductive{nat}} \& \coqref{equations intro.vector}{\coqdocinductive{vector}} \coqdocvariable{A} \coqdocvariable{n}\}. The default obligation
      tactic defined in \coqdoclibrary{Equations.Init} includes a proof-search
      for \coqdocvar{subterm} proofs which can resolve the recursive call obligation
      automatically in this case. \begin{coqdoccode}
\coqdocemptyline
\coqdocindent{1.00em}
\coqdockw{Equations} \coqdef{equations intro.UnzipVect.unzip}{unzip}{\coqdocdefinition{unzip}} \{\coqdocvar{n}\} (\coqdocvar{v} : \coqref{equations intro.vector}{\coqdocinductive{vector}} (\coqref{equations intro.UnzipVect.A}{\coqdocaxiom{A}} \coqexternalref{::type scope:x '*' x}{http://coq.inria.fr/distrib/V8.10.1/stdlib//Coq.Init.Datatypes}{\coqdocnotation{\ensuremath{\times}}} \coqref{equations intro.UnzipVect.B}{\coqdocaxiom{B}}) \coqdocvariable{n}) : \coqref{equations intro.vector}{\coqdocinductive{vector}} \coqref{equations intro.UnzipVect.A}{\coqdocaxiom{A}} \coqdocvar{n} \coqexternalref{::type scope:x '*' x}{http://coq.inria.fr/distrib/V8.10.1/stdlib//Coq.Init.Datatypes}{\coqdocnotation{\ensuremath{\times}}} \coqref{equations intro.vector}{\coqdocinductive{vector}} \coqref{equations intro.UnzipVect.B}{\coqdocaxiom{B}} \coqdocvar{n}\coqdoceol
\coqdocindent{2.00em}
\coqdoctac{by} \coqdockw{wf} (\coqdocdefinition{signature\_pack} \coqdocvar{v}) (@\coqref{equations intro.vector subterm}{\coqdocdefinition{vector\_subterm}} (\coqref{equations intro.UnzipVect.A}{\coqdocaxiom{A}} \coqexternalref{::type scope:x '*' x}{http://coq.inria.fr/distrib/V8.10.1/stdlib//Coq.Init.Datatypes}{\coqdocnotation{\ensuremath{\times}}} \coqref{equations intro.UnzipVect.B}{\coqdocaxiom{B}})) :=\coqdoceol
\coqdocindent{1.00em}
\coqref{equations intro.unzip}{\coqdocdefinition{unzip}} \coqref{equations intro.Vnil}{\coqdocconstructor{Vnil}} := \coqexternalref{::core scope:'(' x ',' x ',' '..' ',' x ')'}{http://coq.inria.fr/distrib/V8.10.1/stdlib//Coq.Init.Datatypes}{\coqdocnotation{(}}\coqref{equations intro.Vnil}{\coqdocconstructor{Vnil}}\coqexternalref{::core scope:'(' x ',' x ',' '..' ',' x ')'}{http://coq.inria.fr/distrib/V8.10.1/stdlib//Coq.Init.Datatypes}{\coqdocnotation{,}} \coqref{equations intro.Vnil}{\coqdocconstructor{Vnil}}\coqexternalref{::core scope:'(' x ',' x ',' '..' ',' x ')'}{http://coq.inria.fr/distrib/V8.10.1/stdlib//Coq.Init.Datatypes}{\coqdocnotation{)}} ;\coqdoceol
\coqdocindent{1.00em}
\coqref{equations intro.unzip}{\coqdocdefinition{unzip}} (\coqref{equations intro.Vcons}{\coqdocconstructor{Vcons}} \coqref{equations intro.Vcons}{\coqdocconstructor{(}}\coqexternalref{pair}{http://coq.inria.fr/distrib/V8.10.1/stdlib//Coq.Init.Datatypes}{\coqdocconstructor{pair}} \coqdocvariable{x} \coqdocvariable{y}\coqref{equations intro.Vcons}{\coqdocconstructor{)}} \coqdocvariable{v}) \coqdockw{with} \coqref{equations intro.unzip}{\coqdocdefinition{unzip}} \coqdocvar{v} := \{\coqdoceol
\coqdocindent{1.00em}
\ensuremath{|} \coqexternalref{pair}{http://coq.inria.fr/distrib/V8.10.1/stdlib//Coq.Init.Datatypes}{\coqdocconstructor{pair}} \coqdocvariable{xs} \coqdocvariable{ys} := \coqexternalref{::core scope:'(' x ',' x ',' '..' ',' x ')'}{http://coq.inria.fr/distrib/V8.10.1/stdlib//Coq.Init.Datatypes}{\coqdocnotation{(}}\coqref{equations intro.Vcons}{\coqdocconstructor{Vcons}} \coqdocvar{x} \coqdocvar{xs}\coqexternalref{::core scope:'(' x ',' x ',' '..' ',' x ')'}{http://coq.inria.fr/distrib/V8.10.1/stdlib//Coq.Init.Datatypes}{\coqdocnotation{,}} \coqref{equations intro.Vcons}{\coqdocconstructor{Vcons}} \coqdocvar{y} \coqdocvar{ys}\coqexternalref{::core scope:'(' x ',' x ',' '..' ',' x ')'}{http://coq.inria.fr/distrib/V8.10.1/stdlib//Coq.Init.Datatypes}{\coqdocnotation{)}} \}.\coqdoceol
\coqdocemptyline
\coqdocnoindent
\coqdockw{End} \coqref{equations intro.UnzipVect}{\coqdocmodule{UnzipVect}}.\coqdoceol
\coqdocemptyline
\end{coqdoccode}
For the diagonal, it is easier to give \coqdocvariable{n} as the decreasing argument
    of the function, even if the pattern-matching itself is on vectors: \begin{coqdoccode}
\coqdocemptyline
\coqdocnoindent
\coqdockw{Equations} \coqdef{equations intro.diag'}{diag'}{\coqdocdefinition{diag'}} \{\coqdocvar{A} \coqdocvar{n}\} (\coqdocvar{v} : \coqref{equations intro.vector}{\coqdocinductive{vector}} (\coqref{equations intro.vector}{\coqdocinductive{vector}} \coqdocvariable{A} \coqdocvariable{n}) \coqdocvariable{n}) : \coqref{equations intro.vector}{\coqdocinductive{vector}} \coqdocvar{A} \coqdocvar{n} \coqdoctac{by} \coqdockw{wf} \coqdocvar{n} :=\coqdoceol
\coqdocnoindent
\coqref{equations intro.diag'}{\coqdocdefinition{diag'}} \coqref{equations intro.Vnil}{\coqdocconstructor{Vnil}} := \coqref{equations intro.Vnil}{\coqdocconstructor{Vnil}} ;\coqdoceol
\coqdocnoindent
\coqref{equations intro.diag'}{\coqdocdefinition{diag'}} (\coqref{equations intro.Vcons}{\coqdocconstructor{Vcons}} \coqref{equations intro.Vcons}{\coqdocconstructor{(}}\coqref{equations intro.Vcons}{\coqdocconstructor{Vcons}} \coqdocvariable{a} \coqdocvariable{v}\coqref{equations intro.Vcons}{\coqdocconstructor{)}} \coqdocvariable{v'}) :=\coqdoceol
\coqdocindent{1.00em}
\coqref{equations intro.Vcons}{\coqdocconstructor{Vcons}} \coqdocvar{a} (\coqref{equations intro.diag'}{\coqdocdefinition{diag'}} (\coqref{equations intro.vmap}{\coqdocdefinition{vmap}} \coqref{equations intro.vtail}{\coqdocdefinition{vtail}} \coqdocvar{v'})).\coqdoceol
\coqdocemptyline
\end{coqdoccode}
One can check using \coqdockw{Extraction} \coqref{equations intro.diag'}{\coqdocdefinition{diag'}} that the computational behavior of \coqref{equations intro.diag'}{\coqdocdefinition{diag'}}
    is indeed not dependent on the index \coqdocvariable{n}. 

\subsubsection{Pattern-matching and axiom K}



 To use the K axiom or UIP with \coqdockw{Equations}, one \textit{must} first set an option
    allowing its use during dependenet pattern-matching compilation. \begin{coqdoccode}
\coqdocemptyline
\coqdocnoindent
\coqdockw{Module} \coqdef{equations intro.KAxiom}{KAxiom}{\coqdocmodule{KAxiom}}.\coqdoceol
\coqdocemptyline
\end{coqdoccode}
By default we disallow the user of UIP, but it can be set. \begin{coqdoccode}
\coqdocemptyline
\coqdocindent{1.00em}
\coqdockw{Set} \coqdockw{Equations} \coqdocvar{With} \coqdocvar{UIP}.\coqdoceol
\coqdocemptyline
\coqdocindent{1.00em}
\coqdockw{Module} \coqdef{equations intro.KAxiom.WithAx}{WithAx}{\coqdocmodule{WithAx}}.\coqdoceol
\coqdocemptyline
\end{coqdoccode}
The user must declare this axiom itself, as an instance of the \coqdocclass{UIP} class. \begin{coqdoccode}
\coqdocemptyline
\coqdocindent{2.00em}
\coqdockw{Axiom} \coqdef{equations intro.KAxiom.WithAx.uipa}{uipa}{\coqdocaxiom{uipa}} : \coqdockw{\ensuremath{\forall}} \coqdocvar{A}, \coqdocclass{UIP} \coqdocvariable{A}.\coqdoceol
\coqdocindent{2.00em}
\coqdockw{Local} \coqdocvar{Existing} \coqdockw{Instance} \coqdocvar{uipa}.\coqdoceol
\coqdocemptyline
\end{coqdoccode}
In this case the following definition uses the \coqdocclass{UIP} axiom just declared. \begin{coqdoccode}
\coqdocemptyline
\coqdocindent{2.00em}
\coqdockw{Equations} \coqdef{equations intro.KAxiom.WithAx.K}{K}{\coqdocdefinition{K}} \{\coqdocvar{A}\} (\coqdocvar{x} : \coqdocvariable{A}) (\coqdocvar{P} : \coqdocvariable{x} \coqexternalref{::type scope:x '=' x}{http://coq.inria.fr/distrib/V8.10.1/stdlib//Coq.Init.Logic}{\coqdocnotation{=}} \coqdocvariable{x} \coqexternalref{::type scope:x '->' x}{http://coq.inria.fr/distrib/V8.10.1/stdlib//Coq.Init.Logic}{\coqdocnotation{\ensuremath{\rightarrow}}} \coqdockw{Type}) (\coqdocvar{p} : \coqdocvariable{P} \coqexternalref{eq refl}{http://coq.inria.fr/distrib/V8.10.1/stdlib//Coq.Init.Logic}{\coqdocconstructor{eq\_refl}})\coqdoceol
\coqdocindent{7.00em}
(\coqdocvar{H} : \coqdocvariable{x} \coqexternalref{::type scope:x '=' x}{http://coq.inria.fr/distrib/V8.10.1/stdlib//Coq.Init.Logic}{\coqdocnotation{=}} \coqdocvariable{x}) : \coqdocvar{P} \coqdocvar{H} :=\coqdoceol
\coqdocindent{3.00em}
\coqref{equations intro.K}{\coqdocdefinition{K}} \coqdocvariable{x} \coqdocvariable{P} \coqdocvariable{p} \coqexternalref{eq refl}{http://coq.inria.fr/distrib/V8.10.1/stdlib//Coq.Init.Logic}{\coqdocconstructor{eq\_refl}} := \coqdocvar{p}.\coqdoceol
\coqdocemptyline
\coqdocindent{1.00em}
\coqdockw{End} \coqref{equations intro.KAxiom.WithAx}{\coqdocmodule{WithAx}}.\coqdoceol
\coqdocemptyline
\end{coqdoccode}
Note that the definition loses its computational content: it will
      get stuck on an axiom. We hence do not recommend its use.


      Equations allows however to use constructive proofs of UIP for types
      enjoying decidable equality. The following example relies on an
      instance of the \coqdocvar{EqDec} typeclass for natural numbers, from which
      we can automatically derive a \coqdocclass{UIP} \coqexternalref{nat}{http://coq.inria.fr/distrib/V8.10.1/stdlib//Coq.Init.Datatypes}{\coqdocinductive{nat}} instance.  Note that
      the computational behavior of this definition on open terms is not
      to reduce to \coqdocvariable{p} but pattern-matches on the decidable equality
      proof.  However the defining equation still holds as a
      \textit{propositional} equality, and the definition of K' is axiom-free. \begin{coqdoccode}
\coqdocemptyline
\coqdocindent{1.00em}
\coqdockw{Equations} \coqdef{equations intro.KAxiom.K'}{K'}{\coqdocdefinition{K'}} (\coqdocvar{x} : \coqexternalref{nat}{http://coq.inria.fr/distrib/V8.10.1/stdlib//Coq.Init.Datatypes}{\coqdocinductive{nat}}) (\coqdocvar{P} : \coqdocvariable{x} \coqexternalref{::type scope:x '=' x}{http://coq.inria.fr/distrib/V8.10.1/stdlib//Coq.Init.Logic}{\coqdocnotation{=}} \coqdocvariable{x} \coqexternalref{::type scope:x '->' x}{http://coq.inria.fr/distrib/V8.10.1/stdlib//Coq.Init.Logic}{\coqdocnotation{\ensuremath{\rightarrow}}} \coqdockw{Type}) (\coqdocvar{p} : \coqdocvariable{P} \coqexternalref{eq refl}{http://coq.inria.fr/distrib/V8.10.1/stdlib//Coq.Init.Logic}{\coqdocconstructor{eq\_refl}})\coqdoceol
\coqdocindent{6.00em}
(\coqdocvar{H} : \coqdocvariable{x} \coqexternalref{::type scope:x '=' x}{http://coq.inria.fr/distrib/V8.10.1/stdlib//Coq.Init.Logic}{\coqdocnotation{=}} \coqdocvariable{x}) : \coqdocvar{P} \coqdocvar{H} :=\coqdoceol
\coqdocindent{2.00em}
\coqref{equations intro.K'}{\coqdocdefinition{K'}} \coqdocvariable{x} \coqdocvariable{P} \coqdocvariable{p} \coqexternalref{eq refl}{http://coq.inria.fr/distrib/V8.10.1/stdlib//Coq.Init.Logic}{\coqdocconstructor{eq\_refl}} := \coqdocvar{p}.\coqdoceol
\coqdocemptyline
\coqdocindent{1.00em}
\coqdockw{Print} \coqdockw{Assumptions} \coqref{equations intro.KAxiom.K'}{\coqdocdefinition{K'}}.\coqdoceol
\coqdocindent{1.00em}
\begin{coqdoccomment}
\coqdocindent{0.50em}
Closed\coqdocindent{0.50em}
under\coqdocindent{0.50em}
the\coqdocindent{0.50em}
global\coqdocindent{0.50em}
context\coqdocindent{0.50em}
\end{coqdoccomment}
\coqdoceol
\coqdocemptyline
\coqdocnoindent
\coqdockw{End} \coqref{equations intro.KAxiom}{\coqdocmodule{KAxiom}}.\coqdoceol
\end{coqdoccode}


\paragraph{Going further}

More examples are available at \url{http://mattam82.github.io/Coq-Equations/examples}

\bibliography{biblio}
\addcontentsline{toc}{chapter}{Bibliography}
\label{cha:bibliography}
\bibliographystyle{myplainnat}
\end{document}

